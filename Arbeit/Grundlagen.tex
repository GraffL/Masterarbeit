\section{Grundlagen}\label{sec:Grundlagen}

\subsection{Spiele in strategischer Form}

\begin{defn}
	Ein \emph{Spiel $\Gamma$ in strategischer Form} ist ein Tupel $(I, X = (X_i)_{i \in I}, (K_i)_{i\in I}, (c_i)_{i\in I})$. Dabei ist
	\begin{itemize}
		\item $I$ die Menge der Spieler,
		\item $X_i$ die Menge der (reinen) Strategien von Spieler $i$,
		\item $(K_i, \cordleq)$, eine total geordnete Menge, der Kostenraum von Spieler $i$ und
		\item $c_i: X_i \to K_i$ die Kostenfunktion von Spieler $i$.
	\end{itemize}
	Das eigentliche Spiel besteht nun daraus, dass jeder Spieler versucht durch die Wahl seiner Strategie die eigenen Kosten zu minimieren.
	
	Wir nennen ein solches Spiel \emph{endlich}, wenn der gesamte Strategieraum $X$ endlich ist.
\end{defn}

\begin{beob}
	Klassische Kostenminimierungsspiele erhält man durch Wahl von $(\IR_{\geq 0}, \leq)$ als Kostenraum für alle Spieler, Nutzenmaximierungsspiele durch Wahl von $(\IR, \geq)$ als \glqq Kosten\grqq raum.
\end{beob}

\begin{beob}
	Ist ein Spiel endlich, so können wir in der Regel ohne Einschränkung annehmen, dass auch die Menge der Spieler endlich ist. Denn die Menge $X = \prod_{i\in I} X_i$ kann nur endlich sein, wenn höchsten endliche viele $X_i$ einelementig sind. Also haben in einem endlichen Spiel nur endlich viele Spieler mehr als eine Strategie. Für die Suche beispielsweise nach Nash-Gleichgewichten oder Verbesserungspfaden spielen aber nur solche Spieler eine Rolle. \todo{Kann man diese Erkenntnis irgendwie mit Morphismen formalisieren (evtl. über Retrakte?)}
\end{beob}

\begin{notation}
	Zu einem festen Spieler $i$ bezeichne $X_{-i} := \prod_{j \in I\backslash\{i\}} X_j$ das Produkt aller Strategieräume außer dem von Spieler $i$. Zu jedem Strategieprofil\todo{Sollte der Begriff Strategieprofil eigens definiert werden?}{} $x \in X$ bezeichne dann $x_{-i}$ die Projektion dieses Profils auf den Raum $X_{-i}$ und $x_i$ die Projektion auf $X_i$ (also die von Spieler $i$ gewählte Strategie). Wir schreiben dann auch $(x_i, x_{-i})$ für das Strategieprofil $x$.
	
	Später werden wir auch Abbildungen $\phi_i: X_i \to Y_i$ zwischen den spielerspezifischen Strategieräumen betrachten und die durch diese induzierte Abbildung $X \to Y: x = (x_i)_{i \in I} \mapsto \phi(x) := (\phi_i(x_i))_{i \in I}$ mit $\phi$ bezeichnen. Analog zur Notation für Strategieräume werden wir außerdem die Notation $\phi_{-i}: X_{-i} \to Y_{-i}$ verwenden.
\end{notation}

\begin{defn}
	Ein Strategieprofil $x \in X$ ist ein \emph{Nash-Gleichgewicht}, wenn für jeden Spieler $i \in I$ und jede seiner Strategien $\hat{x}_i$ gilt:
		\[c_i(\hat{x}_i, x_{-i}) \cordgeq c_i(x)\]
	Ein Nash-Gleichgewicht ist also ein Strategieprofil, aus dem heraus kein Spieler einen Anreiz für einen einseitigen Strategiewechsel hat.	
\end{defn}

Aus \cite{MonShap}:

\begin{defn}
	Eine Folge von Strategieprofilen $x^0, x^1, x^2, \dots$ ist ein \emph{Verbesserungspfad}, wenn folgende beiden Eigenschaften erfüllt sind:
	\begin{enumerate}
		\item Für jede Stelle $n$ gibt es einen Spieler $i(n) \in I$, sodass das Profil $x^{n+1}$ aus $x^n$ durch alleinige Abweichung dieses Spielers entsteht, d.h. $x^{n+1} = (x^{n+1}_{i(n)}, x^n_{-i(n)})$
		\item Der abweichende Spieler $i(n)$ verbessert sich, d.h. $c_{i(n)}(x^{n+1}) \cordle c_{i(n)}(x^n)$.
	\end{enumerate}
	Wir nennen einen endlichen Verbesserungspfad $x^0, x^1, \dots, x^n$ \emph{abgeschlossen}, wenn er nicht mehr nach hinten verlängert werden kann, d.h. es keine Strategie $\hat{x}_i$ gibt, mit $c_{i}(x^n_{-i}, \hat{x}_i) \cordle c_{i}(x^n)$.
\end{defn}

\begin{defn}
	Ein Spiel $\Gamma$ hat die \emph{finite improvement property (FIP)}, wenn jeder Verbesserungspfad endlich ist.
\end{defn}

\begin{beob}\label{beob:VerbPfadeundNGe}
	Jedes Ende eines abgeschlossenen Verbesserungspfades ist ein Nash-Gleichgewicht. Denn wäre dem nicht so, dann gäbe es wenigstens einen Spieler, der sich durch Abweichen noch verbessern kann - was zu einer Verlängerung des Verbesserungspfades führen würde. 
		
	Umgekehrt ist offenkundig auch jedes Nash-Gleichgewicht Ende wenigstens eines abgeschlossenen Verbesserungspfades - nämlich des trivialen, nur aus diesem Strategieprofil bestehenden Verbesserungspfades
\end{beob}

\begin{kor}\label{kor:ExVerbPfadExNG}\todo{Sollte eine Beobachtung wirklich ein Korrolar haben?}
	Ein Spiel $\Gamma$ besitzt genau dann (mindestens) ein Nash-Gleichgewicht, wenn es (mindestens) einen endlichen, maximalen Verbesserungspfad besitzt. Ein Spiel mit FIP besitzt dementsprechend immer wenigstens ein Nash-Gleichgewicht.
\end{kor}

Aus \cite{BestRespPot}

\begin{defn}
	Ein Verbesserungspfad $x^0, x^1, x^2, \dots$ heißt \emph{Beste Antwort-Pfad}\todo{Voorneveld nennt es (ohne Verbesserungsbedinung) beste Antwort-kompatibel}, wenn der abweichende Spieler jeweils eine beste Alternativstrategie wählt, d.h. 
		\[c_{i(n)}(x^{n+1}) = \min_{\hat{x}_{i(n)} \in X_{i(n)}} c_{i(n)}(\hat{x}_i,x^n_{-i(n)}).\]
	\todo{Insbesondere setzt das natürlich voraus, dass ein solches Maximum auf dem Pfad immer existiert.}
\end{defn}


\subsection{Auslastungsspiele}

\begin{defn}\label{def:Auslastungsmodel}
	Ein \emph{Auslastungsmodell $M$} ist gegeben durch ein Tupel $(I, R, (S_i)_{i\in I}, K, (g_r)_{r \in R})$. Dabei ist
	\begin{itemize}
		\item $I$ die Menge der Spieler,
		\item $R$ die Menge der zur Verfügung stehenden Ressourcen,
		\item $S_i \subseteq \PSet(R)$ die Menge von endlichen Teilmengen der Ressourcenmenge, unter denen sich der Spieler $i$ für eine Teilmenge entscheiden kann,
		\item $K$, eine geordnete abelsche Gruppe, der Kostenraum der Ressourcen und
		\item $g_r: \IR \to K$ eine Funktion, welche die Kosten der Ressource $r \in R$ in Abhängigkeit von ihrer Auslastung beschreibt.
	\end{itemize}
\end{defn}

\begin{defn}\label{def:Auslastungsspiel}
	Jedes Auslastungsmodell $M$ induziert ein \emph{Auslastungsspiel} $\Gamma(M) := (I, S = (S_i)_{i\in I}, (K)_{i\in I}, (c_i)_{i \in I})$ durch die Kostenfunktion:
	\[c_i: S \to K: s \mapsto \sum_{r \in R} g_r(l_r(s)) \]
	wobei $l_r: S \to \IN: s \mapsto \abs{\{i\in I \mid r \in s_i\}}$ die \emph{Lastfunktion}\todo{Hier bekommt man wohl Probleme, wenn Spieler- bzw. Ressourcenmenge unendlich ist.}{} der Ressource ist.
\end{defn}

\begin{bsp}.
	
	\todo[inline]{Netzwerkauslastungsspiel}
\end{bsp}

\begin{defn}\label{def:gewAuslastungsspiel}
	Zusammen mit einem Gewichtsvektor $(w_i)_{i\in I}$ induziert ein Auslastungsmodell $M$ 
	\begin{itemize}
		\item ein \emph{kostengewichtetes Auslastungsspiel} $\Gamma_c(M, w) := (I, S, (K)_{i\in I}, (c_i)_{i \in I})$ durch die Kostenfunktion:
		\[c_i: S \to K: s \mapsto \sum_{r \in R} w_i\cdot g_r(l_r(s)) \]
		und die gleiche Lastfunktion wie im ungewichteten Fall.
		\item ein \emph{lastgewichtetes Auslastungsspiel} $\Gamma_l(M, w) := (I, S, (K)_{i\in I}, (c_i)_{i \in I})$ durch die Kostenfunktion wie im ungewichteten Fall und die Lastfunktion $l_r: S \to \IN: s \mapsto \sum\{w_i \mid r \in s_i\}$.
		\item ein \emph{gewichtetes Auslastungsspiel} $\Gamma(M, w) := (I, S, (K)_{i\in I}, (c_i)_{i \in I})$ durch die Kostenfunktion:
		\[c_i: S \to K: s \mapsto \sum_{r \in R} w_i\cdot g_r(l_r(x)) \]
		und die Lastfunktion $l_r: S \to \IN: s \mapsto \sum\{w_i \mid r \in s_i\}$.
	\end{itemize}	
\end{defn}

\begin{bsp}.
	
	\todo[inline]{gewichtetes Netzwerkauslastungsspiel}
\end{bsp}

\begin{defn}\label{def:skalierteAuslastungsspiel}
	\todo{Ist diese Definition wirklich sinnvoll?}
	Zusammen mit einer Skalierungsfunktionen $(f_i: K \to K_i)_{i\in I}$ und verallgemeinerten Lastfunktionen $(h_r: X \to \IR)_{r\in R}$ induziert ein Auslastungsmodell $M$ 
	\begin{itemize}
		\item ein \emph{kostenskaliertes Auslastungsspiel} $\Gamma_c(M, f_i) := (I, S, (K)_{i\in I}, (c_i)_{i \in I})$ durch die Kostenfunktion:
		\[c_i: S \to K: s \mapsto \sum_{r \in R} f_i(g_r(l_r(s))) \]
		und die gleiche Lastfunktion wie im ungewichteten Fall.
		\item ein \emph{lastskaliertes Auslastungsspiel} $\Gamma_l(M, h_r) := (I, S, (K)_{i\in I}, (c_i)_{i \in I})$ durch die Kostenfunktion:
		\[c_i: S \to K: s \mapsto \sum_{r \in R} g_r(h_r(s)) \]
		\item ein \emph{verallgemeinertes gewichtetes Auslastungsspiel} $\Gamma(M, f_i, h_r) := (I, S, (K)_{i\in I}, (c_i)_{i \in I})$ durch die Kostenfunktion:
		\[c_i: S \to K: s \mapsto \sum_{r \in R} f_i(g_r(h_r(s))).\]
	\end{itemize}	
\end{defn}