\section{Grundlagen}\label{sec:Grundlagen}

\subsection{Spiele in strategischer Form}

\begin{defn}
	Ein \emph{(nichtkooperatives) Spiel in strategischer Form} ist ein Tupel $\Gamma = (I, X = (X_i)_{i \in I}, (c_i)_{i\in I})$. Dabei ist
	\begin{itemize}
		\item $I$ die Menge der Spieler,
		\item $X_i$ die nicht-leere Menge der (reinen) Strategien von Spieler $i$,
		\item $c_i: X_i \to \IR$ die Kostenfunktion von Spieler $i$.
	\end{itemize}
	Das eigentliche Spiel besteht nun daraus, dass jeder Spieler versucht durch die Wahl einer Strategie $x_i \in X_i$ aus seinem Strateegieraum die eigenen Kosten $c_i((x_j)_{j \in I})$ zu minimieren. Das Tupel $x \coloneqq (x_j)_{j \in I}$ bezeichnen wir dabei als \emph{Strategieprofil}.
	
	Wir nennen ein solches Spiel \emph{endlich}, wenn der gesamte Strategieraum $X$ endlich ist und $N$-Personenspiel, wenn $\abs{I} = N \in \IN$ gilt.
\end{defn}

\begin{beob}\label{beob:endlicheSpiele}
	Ist ein Spiel endlich, so können wir in der Regel ohne Einschränkung annehmen, dass auch die Menge der Spieler endlich ist. Denn die Menge $X = \prod_{i\in I} X_i$ kann nur endlich sein, wenn höchsten endliche viele $X_i$ einelementig sind. Also haben in einem endlichen Spiel nur endlich viele Spieler mehr als eine Strategie. Für die Suche beispielsweise nach Nash-Gleichgewichten oder Verbesserungspfaden spielen aber nur solche Spieler eine Rolle. Wir werden dies später in \Cref{sec:Morphismen} formal zeigen (siehe \Cref{bsp:EinStratSpielerWeglassen}).
\end{beob}

\begin{notation}
	Zu einem Strategieprofil $x \in X$ und einer einzelnen Strategie $\hat{x}_i \in X_i$ bezeichnen wir einer Konvention aus \cite{Polyequilibrium} folgend mit $(x \mid \hat{x}_i)$ das Strategieprofil, das aus $x$ entsteht, wenn Spieler $i$ einseitig seine Strategie von $x_i$ auf $\hat{x}_i$ ändert.
\end{notation}

\begin{defn}
	Ein Strategieprofil $x \in X$ ist ein \emph{Nash-Gleichgewicht}, wenn für jeden Spieler $i \in I$ und jede seiner Strategien $\hat{x}_i$ gilt:
		\[c_i(x \mid \hat{x}_i) \geq c_i(x)\]
	Ein Nash-Gleichgewicht ist also ein Strategieprofil, aus dem heraus kein Spieler einen Anreiz für einen einseitigen Strategiewechsel hat.	
\end{defn}

In \cite[Definition 2.2]{KoordDummy} werden Koordinations- und Dummy-Spiele definiert. Analog dazu definieren wir hier zusätzlich skalierte Koordinationsspiele:
\begin{defn}
	Ein Spiel $\Gamma = (I, X = (X_i)_{i \in I}, (c_i)_{i\in I})$ heißt
	\begin{itemize}
		\item \emph{Koordinationsspiel}, wenn alle Spieler eine gemeinsame Kostenfunktion nutzen, wenn also gilt $c_i = c_j$ für alle Spieler $i,j \in I$.
		\item \emph{skaliertes Koordinationsspiel}, wenn jeder Spieler eine streng monoton skalierte Variante einer gemeinsamen Kostenfunktion verwendet, d.h. wenn es eine Funktion $c: X \to \IR$ und streng monotone Funktionen $f_i: \IR \to \IR$ gibt, sodass für jeden Spieler $c_i = f_i \circ c$ gilt.
		\item \emph{Dummyspiel}, wenn die alleinige Abweichung eines einzelnen Spielers von einem gegebenen Strategieprofil nie zu einer Änderung seiner Kosten führt. D.h. wenn für alle $x \in X, i \in I$ und $\hat{x}_i \in X_i$ gilt $c_i(x) = c_i(x \mid \hat{x}_i)$.
	\end{itemize}
\end{defn}

\begin{beob}
	In einem (skalierten) Koordinationsspiel ist ein Strategieprofil bereits ein Nash-Gleichgewicht, wenn es nur für einen der Spieler optimal ist. Insbesondere haben damit endliche (skalierte) Koordinationsspiele immer mindestens ein Nash-Gleichgewicht, nämlich das (globale) Minimum der gemeinsamen Kostenfunktion.
	
	In einem Dummy-Spiel ist \emph{jedes} Strategieprofil ein Nash-Gleichgewicht, da ein einzelner Spieler durch einseitige Abweichung seine Kosten nie verändern und daher insbesondere nicht verringern kann.
\end{beob}


\subsection{Abweichungspfade}

Aus \cite{MonShap}/\cite{BestRespPot}\todo{Motivation/Anschauung}

\begin{defn}
	Eine Folge von Strategieprofilen $\gamma = (x^0, x^1, x^2, \dots) \in X^\IN$ ist ein \emph{Pfad}, wenn jeder Schritt aus der Abweichung eines einzelnen Spielers besteht. Das heißt wenn es für jeden Schritt $k$ einen Spieler $i(k) \in I$ gibt, sodass das Profil $x^{k}$ aus $x^{k-1}$ durch alleinige Abweichung dieses Spielers entsteht, also $x^{k} = \left(x^{k-1} \middle| x^{k}_{i(k)}\right)$.
	
	Ein solcher Pfad heißt
	\begin{itemize}
		\item \emph{Nichtverschlechterungspfad}, wenn sich der abweichende Spieler jeweils nicht verschlechtert. Das heißt es gilt in jedem Schritt: $c_{i(k)}(x^{k}) \leq c_{i(k)}(x^{k-1})$.
		\item \emph{schwacher Verbesserungspfad}, wenn er ein Nichtverschlechterungspfad ist und sich an mindestens einer Stelle der abweichende Spieler echt verbessert. Das heißt es gibt einen Schritt $k$ mit $c_{i(k)}(x^k) < c_{i(k)}(x^{k-1})$.
		\item \emph{Verbesserungspfad}, wenn sich der abweichende Spieler jeweils echt verbessert. Das heißt es gilt in \emph{jedem} Schritt: $c_{i(k)}(x^k) < c_{i(k)}(x^{k-1})$.
		\item \emph{Beste-Antwort-Pfad}\todo{Voorneveld nennt es beste Antwort-kompatibel}, wenn der abweichende Spieler jeweils eine beste Alternativstrategie wählt, d.h. $c_{i(k)}(x^k) = \min_{\hat{x}_{i(k)} \in X_{i(k)}} c_{i(k)}(x^{k-1} \mid \hat{x}_i)$.
		\item \emph{schwacher Beste-Antwort-Verbesserungspfad}, wenn er ein schwacher Verbesserungspfad und ein Beste-Antwort-Pfad ist.
		\item \emph{Beste-Antwort-Verbesserungspfad}, wenn er ein Verbesserungspfad und ein Beste-Antwort-Pfad ist.
	\end{itemize}
	
	Für einen endlichen Pfad $\gamma = (x^0, \dots, x^n)$ ist
	\begin{itemize}
		\item die \emph{Gesamtänderung} entlang des Pfades $\PfadAend(\gamma)$ definiert als die Summe aller Änderungen für die jeweils abweichenden Spieler:
		\[\PfadAend(\gamma) \coloneqq \sum_{k=1}^{n} \left(c_{i(k)}(x^k) - c_{i(k)}(x^{k-1})\right) \]
		\item $\gamma$ ein \emph{abgeschlossener Verbesserungspfad}, wenn er ein Verbesserungspfad ist, der nicht mehr nach hinten verlängert werden kann, d.h. es keine Strategie $\hat{x}_i$ gibt, mit $c_{i}(x^n \mid \hat{x}_i) < c_{i}(x^n)$.\todo{Schöner formulieren}
		\item $\gamma$ ein \emph{$n$-Zykel}, wenn $x^0 = x^n$ gilt.
		\item $\invPf{\gamma} \coloneqq (x^n, x^{n-1}, \dots, x^0)$ der in die andere Richtung durchlaufende Pfad.
	\end{itemize}

	Ist zusätzlich $\mu = (y^0, y^1, \dots)$ ein beliebiger Pfad, so ist $\gamma\Pfadkomp\mu \coloneqq (x^0, \dots, x^n, y^0, y^1, \dots)$ die Verknüpfung der beiden Pfade (und offensichtlich selbst wieder ein Pfad).
\end{defn}

\begin{beob}\todo{Ordnen/Aussortieren/Beweise?}
	\begin{itemize}
		\item Ein endlicher Nichtverschlechterungspfad $\gamma$ ist genau dann ein schwacher Verbesserungspfad, wenn $\invPf{\gamma}$ kein Nichtverschlechterungspfad ist. 
		
		\item Zwei Pfade $\gamma$ und $\mu$ von $x$ nach $y$ haben genau dann die gleiche Gesamtänderung, wenn $\gamma\Pfadkomp\invPf{\mu}$ eine Gesamtänderung von $0$ hat. Es gilt also für zwei solche Pfade: 
		\[\PfadAend(\gamma) = \PfadAend(\mu) \iff \PfadAend(\gamma\Pfadkomp\invPf{\mu})=0 \]
		
		\item Jeder Beste-Antwort-Pfad ist automatisch ein Nichtverschlechterungspfad.
	\end{itemize}
\end{beob}

\begin{defn}
	Eine Teilmenge $Y \subseteq X$ des Strategieraumes bezeichnen wir als \emph{Pfadzusammenhangskomponente}, wenn es für je zwei $x, y \in Y$ einen Pfad von $x$ nach $y$ in $Y$ gibt. Zu einem gegebeenen Strategieprofil $x \in X$ ist \emph{die Pfadzusammenhangskomponente von $x$} die (bezüglich Inklusion) maximale Pfadzusammenhangskomponente $Y_x \subseteq X$, welche $x$ enthält.\todo{Wohldefiniertheit? Evtl. direkt mit Relationen definieren?}
\end{defn}

\begin{beob}
	Die Pfadzusammenhangskomponente eines Strategieprofils $x$ besteht aus allen Strategieprofilen, die sich an höchstens endlich vielen Stellen von $x$ unterscheiden, d.h.
		\[Y_x = \left\{y \in X \mid \exists J \subseteq I: \abs{J} < \infty \text{ und } \forall i \in I\setminus J: x_i = y_i \right\} \]
	Insbesondere ist also für $N$-Personenspiele der gesamte Strategieraum eine einzige Zusammenhangskomponente.
\end{beob}

\begin{defn}
	Ein Spiel $\Gamma$ hat die \emph{finite improvement property (FIP)}, wenn jeder Verbesserungspfad endlich ist.
\end{defn}

Eine anschauliche Bedeutung der FIP ist die, dass der natürliche Algorithmus zum Finden von Nash-Gleichgewichten immer terminiert: Dazu startet man mit einem beliebigen Strategieprofil und führt, solange dieses noch kein Nash-Gleichgewicht ist, einen der dann zur Auswahl stehenden Verbesserungsschritte durch. Im Laufe dieses Verfahrens beschreitet man gerade einen Verbesserungspfad und das Verfahren terminiert genau dann immer, wenn alle Verbesserungspfade endlich sind.

\begin{beob}\label{beob:VerbPfadeundNGe}
	Das Ende eines abgeschlossenen Verbesserungspfades ist immer ein Nash-Gleichgewicht. Denn wäre dem nicht so, dann gäbe es wenigstens einen Spieler, der sich durch Abweichen noch verbessern kann - was zu einer Verlängerung des Verbesserungspfades führen würde. 
		
	Umgekehrt ist offenkundig auch jedes Nash-Gleichgewicht Ende wenigstens eines abgeschlossenen Verbesserungspfades (sogar Beste-Antwort-Verbesserungspfades) - nämlich des trivialen, nur aus diesem Strategieprofil bestehenden Verbesserungspfades.
\end{beob}

\begin{kor}\label{kor:ExVerbPfadExNG}
	Ein Spiel $\Gamma$ besitzt genau dann (mindestens) ein Nash-Gleichgewicht, wenn es (mindestens) einen endlichen, maximalen (Beste-Antwort-)Verbesserungspfad besitzt. Ein Spiel mit FIP besitzt dementsprechend immer wenigstens ein Nash-Gleichgewicht.
\end{kor}


\subsection{Auslastungsspiele}

Eine Klasse von Spielen, welche (im endlichen Fall) immer ein Nash-Gleichgewicht besitzt, bilden die sogenannten Auslastungsspiele, welche erstmals von \citeauthor{RosenthalPotential} in \cite{RosenthalPotential} definiert wurden. Im folgenden wollen wir diese Klasse sowie einige Varianten davon definieren. Verschiedene Eigenschaften dieser sowie insbesondere ihre Beziehung zu Potentialspielen werden wir später in \Cref{sec:Auslastungsspiele} diskutieren.

\begin{defn}\label{def:Auslastungsmodel}
	Ein \emph{Auslastungsmodell $M$} ist gegeben durch ein Tupel $(I, R, (S_i)_{i\in I}, (g_r)_{r \in R})$. Dabei ist
	\begin{itemize}
		\item $I$ die Menge der Spieler,
		\item $R$ die Menge der zur Verfügung stehenden Ressourcen,
		\item $g_r: \IR_{\geq 0} \to \IR$ eine Funktion, welche die Kosten der Ressource $r \in R$ in Abhängigkeit von ihrer Auslastung beschreibt, und
		\item $S_i \subseteq \PSet(R)$ die Menge von Teilmengen der Ressourcenmenge, unter denen sich der Spieler $i$ für eine Teilmenge entscheiden kann. 
	\end{itemize}
\end{defn}

\begin{defn}\label{def:Auslastungsspiel}
	Ein Auslastungsmodell $M$ induziert ein \emph{Auslastungsspiel} $\Gamma(M) := (I, S = (S_i)_{i\in I}, (c_i)_{i \in I})$ durch die Kostenfunktion:
	\[c_i: S \to \IR: s \mapsto \sum_{r \in R} g_r(l_r(s)) \]
	wobei $l_r: S \to \IN: s \mapsto \abs{\{i\in I \mid r \in s_i\}}$ die der Ressource ist, wenn alle $c_i$ und alle $l_r$ auf ganz $S$ wohldefiniert sind (also die entsprechenden Summen für jedes $s \in S$ konvergieren).
\end{defn}

\begin{bem}
	Eine Lastfunktion $l_r$ ist genau dann\todo{Wirklich \glqq genau dann wenn\grqq ? Dann evtl. in Definition aufnehmen.}{} wohldefiniert, wenn diese Ressource $r$ von höchstens endlich vielen verschiedenen Spielen benutzt werden kann. Das heißt wenn gilt: $\abs{\Set{i \in I | \exists s_i \in S_i: r \in s}} < \infty$. Sind alle $l_r$ bereits wohldefiniert, so ist eine hinreichende Bedingung für die Wohldefiniertheit von $c_i$, dass alle diesem Spieler zur Auswahl stehende Teilmengen von $R$ endlich sind. Diese Bedingung ist aber nicht notwendig.
	
	Insbesondere induzieren Auslastungsmodelle mit endlicher Ressourcen- und Spielerzahl immer ein Auslastungsspiel.
	
	Ist die Ressourcenmenge $R$ endlich, so ist das induzierte Auslastungsspiel ein endliches Spiel (d.h. dessen Strategieraum ist endlich). Ist umgekehrt der Strategieraum endlich, so können wir ohne Einschränkung davon ausgehen, dass auch $R$ endlich ist. Denn in diesem Fall tauchen nur endlich viele der Ressourcen in einer Strategie wenigstens eines der Spieler auf - und alle anderen Ressourcen können für die Betrachtung des Spiels ignoriert werden.
\end{bem}

\begin{bsp}
	Ein Beispiel für Auslastungsmodelle sind \emph{Netzwerkauslastungsmodelle}. Diese sind gegeben durch einen (nicht notwendigerweise endlichen) Multigraphen $G=(V,E)$, einer Kostenfunktion $c_e: \IR_{\geq 0} \to \IR$ für jede Kante $e \in E$ und für jeden Spieler $i \in I$ je ein ausgezeichneter Start- und Endknoten $s_i, t_i \in V$. Die Ressourcen sind dann die Kanten des Graphen, die Strategien von Spieler $i \in I$ entsprechen den endlichen $s_i,t_i$-Pfaden in $G$. 
	
	Zusätzlich ist es möglich bestimmte Kanten nur für einen Teil der Spieler zuzulassen. Dadurch wird dann der Strategieraum der anderen Spieler entsprechend eingeschränkt.
\end{bsp}

Verallgemeinerungen von Auslastungsspielen erhalten wir durch das Einführen von Spielergewichten bzw. dem Abändern der Form, in der die Kosten berechnet werden:

\begin{defn}\label{def:gewAuslastungsspiel}
	Zusammen mit einem positiven Gewichtsvektor $(w_i)_{i\in I} \in \IR_{> 0}^I$ induziert ein Auslastungsmodell $M$ (wiederum unter der Voraussetzung, dass alle beteiligten Funktionen wohldefiniert sind)
	\begin{itemize}
		\item ein \emph{kostengewichtetes Auslastungsspiel} $\Gamma_c(M, w) := (I, S, (c_i)_{i \in I})$ durch die Kostenfunktion:
		\[c_i: S \to \IR: s \mapsto \sum_{r \in R} w_i\cdot g_r(l_r(s)) \]
		und die gleiche Lastfunktion wie im ungewichteten Fall.
		\item ein \emph{lastgewichtetes Auslastungsspiel} $\Gamma_l(M, w) := (I, S, (c_i)_{i \in I})$ durch die Kostenfunktion wie im ungewichteten Fall und die Lastfunktion $l_r: S \to \IN: s \mapsto \sum\{w_i \mid r \in s_i\}$.
		\item ein \emph{gewichtetes Auslastungsspiel} $\Gamma(M, w) := (I, S, (c_i)_{i \in I})$ durch die Kostenfunktion:
		\[c_i: S \to \IR: s \mapsto \sum_{r \in R} w_i\cdot g_r(l_r(s)) \]
		und die Lastfunktion $l_r: S \to \IN: s \mapsto \sum\{w_i \mid r \in s_i\}$.
	\end{itemize}	
\end{defn}

\begin{bsp}.
	
	\todo[inline]{gewichtetes Netzwerkauslastungsspiel}
\end{bsp}

Weitere Varianten von Auslastungsspielen:\todo{Quelle?}

\begin{defn}\label{def:weitereAuslastungsspiel}
	\begin{itemize}
		\item Zusammen mit einer streng monotonen Skalierungsfunktionen $(f_i: \IR \to \IR)_{i\in I}$ erhält man aus einem Auslastungsmodell ein \emph{skaliertes Auslastungsspiel} durch die Kostenfunktionen:
			\[c_i: S \to \IR: s \mapsto \sum_{r \in R} f_i(g_r(l_r(s))) \]
		\item Ein \emph{Auslastungsmodell/-spiel mit spielerspezifischen Ressourcenkosten} erhält man, wenn es zu jeder Ressource $r \in R$ und jedem Spieler $i \in I$ eine eigene Funktion $g_r^i: \IR_{\geq 0} \to \IR$ gibt und die Kostenfunktionen der Spieler definiert sind als
			\[c_i: S \to \IR: s \mapsto \sum_{r \in R} g_r^i(l_r(s)) \]
	\end{itemize}
\end{defn}

\begin{bem}
	Jedes skalierte Auslastungsspiel ist insbesondere ein Auslastungsspiel mit spielerspezifischen Ressourcenkosten: Setze dazu $g_r^i \coloneqq f_i \circ g_r$.
\end{bem}

\begin{defn}
	Ein Auslastungsmodell/-spiel heißt \emph{nicht-anonym}, wenn die Kosten einer Ressource nicht nur von der Zahl bzw. dem Gesamtgewicht der sie nutzenden Spieler, sondern auch von deren Identität abhängt. Die Lastfunktionen sind dann also definiert als
		\[l_r: S \to \Pc(I): s \mapsto \Set{i \in I | r \in s_i }\]
	und die Ressourcenkostenfunktionen sind von der Form
		\[g_r: \Pc(I) \to \IR. \]
\end{defn}

Speziellere Klassen von Auslastungsspielen erhalten wir hingegen, wenn die Mengen der zulässigen Teilmengen der einzelnen Spieler zusätzliche Struktur besitzen. Der einfachste Fall hierfür ist, dass jeder Spieler immer nur genau eine Ressource auf einmal wählen kann:

\begin{defn}
	Ein Auslastungsmodell/-spiel heißt \emph{Singelton-Auslastungsmodell/-spiel}, wenn jede Strategie jedes Spielers aus genau einer Ressource besteht.
\end{defn}

Eine andere Möglichkeit ist, dass die zulässigen Teilmengen eines Spielers Basen eines Matroids auf der Ressourcenmenge bilden:

\begin{defn}
	Ein ungewichtetes Auslastungsmodell/-spiel heißt \emph{Matroidmodell/-spiel}, wenn es für jeden Spieler $i \in I$ einen Matroid $M_i$ auf der Ressourcenmenge gibt, sodass $S_i$ gerade die Basen dieses Matroids sind.
\end{defn}

\begin{defn}
	Ein Teilmengensystem $M \subseteq \PSet(R)$ auf einer endlichen Menge $R$ heißt Matroid, wenn es die folgenden drei Axiome erfüllt:\todo{[citation needed(?)] Matroid}
	\begin{enumerate}
		\item $\emptyset \in M$ (oder äquivalent dazu: $M \neq \emptyset$)
		\item $M$ ist abgeschlossen unter Inklusion, d.h. $\forall S \in M, T \subseteq M: T \subseteq S \implies T \in M$.
		\item $\forall S, T \in M: \abs{T} < \abs{S} \implies \exists r \in S \setminus T: T \cup \set{r} \in M$.
	\end{enumerate}
	Die Mengen in $M$ bezeichnen wir dann als \emph{unabhängige Mengen}, die bezüglich Inklusion maximalen Mengen unter diesen nennen wir \emph{Basen des Matroids}.
\end{defn}

Eine wichtige Eigenschaft, welche alle Matroide erfüllen, ist der starke Basisaustauschsatz (vgl. \cite[Satz 3.7]{OptiIVSkript}):

\begin{satz}
	Seien $M$ ein Matroid, $B_1, B_2$ zwei Basen von $M$ und $r \in B_1 \setminus B_2$ beliebig. Dann gibt es ein Element $q \in B_2 \setminus B_1$, sodass das vertauschen dieser beiden Elemente wieder zwei Basen ergibt, d.h. sowohl $B_1 \setminus \set{r} \cup \set{q}$ als auch $B_2 \setminus \set{q} \cup \set{r}$ sind ebenfalls Basen von $M$.
\end{satz}

Mittels Induktion ergibt sich daraus die Basen-Matching-Eigenschaft (vgl. \cite{OptiIVSkript}[Lemma 3.2]):

\begin{kor}\label{kor:BasenMatchingEig}
	Seien $M$ ein Matroid und $B_1, B_2$ zwei Basen von $M$. Dann gibt es eine Partitionierung der symmetrischen Differenz $B_1 \symDiff B_2 \coloneqq B_1 \setminus B_2 \cup B_2 \setminus B_1$ in eine Menge von Tupeln der Form $(q,r)$ mit $q \in B_1\setminus B_2$ und $r \in B_2 \setminus B_1$, sodass $B_1 \setminus \set{q} \cup \set{r}$ ebenfalls eine Basis von $M$ ist. Es gibt also eine Menge von Tupeln $A \subseteq (B_1 \setminus B_2) \times (B_2 \setminus B_1)$, sodass gilt:
	\begin{itemize}
		\item $\forall q \in B_1 \setminus B_2 \exists! r \in B_2 \setminus B_1: (q,r) \in A$
		\item $\forall r \in B_2 \setminus B_1 \exists! q \in B_1 \setminus B_2: (q,r) \in A$
		\item $\forall (q,r) \in A: B_1 \setminus \set{q} \cup \set{r}$ ist eine Basis von $M$
	\end{itemize}
\end{kor}