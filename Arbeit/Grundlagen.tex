\section{Grundlagen}\label{sec:Grundlagen}

\subsection{Spiele in strategischer Form}

\begin{defn}
	Ein \emph{(nichtkooperatives) Spiel in strategischer Form} ist gegeben durch ein Tupel $\Gamma = (I, X = \prod_{i\in I} X_i, (c_i)_{i\in I})$. Dabei ist
	\begin{itemize}
		\item $I$ die Menge der Spieler,
		\item $X_i$ die nicht-leere Menge der (reinen) Strategien von Spieler $i$,
		\item $c_i: X \to \IR$ die Kostenfunktion von Spieler $i$.
	\end{itemize}
	Das eigentliche Spiel besteht nun daraus, dass jeder Spieler $\ib$ versucht durch die Wahl einer Strategie $x_\ib \in X_\ib$ aus seinem Strateegieraum die eigenen Kosten $c_\ib((x_i)_{i \in I})$ zu minimieren. Das Tupel $x \coloneqq (x_i)_{i \in I}$ bezeichnen wir dabei als \emph{Strategieprofil}.
	
	Wir nennen ein solches Spiel \emph{endlich}, wenn der gesamte Strategieraum $X$ endlich ist, und $N$-Personenspiel, wenn $\abs{I} = N \in \IN$ gilt.
\end{defn}

\begin{beob}\label{beob:endlicheSpiele}
	Ist ein Spiel endlich, so können wir in der Regel ohne Einschränkung annehmen, dass auch die Menge der Spieler endlich ist. Denn die Menge $X = \prod_{i\in I} X_i$ kann nur endlich sein, wenn alle bis auf endliche viele $X_i$ einelementig sind. Also haben in einem endlichen Spiel nur endlich viele Spieler mehr als eine Strategie. Für die Suche beispielsweise nach Nash-Gleichgewichten oder Verbesserungspfaden spielen aber nur solche Spieler eine Rolle. Wir werden dies später in \Cref{sec:Morphismen} formal zeigen (siehe \Cref{kor:EinStratSpielerWeglassen}).
\end{beob}

\begin{notation}
	Ein endliches $2$-Personenspiel lässt sich leicht als Matrix darstellen. Deren Zeilen werden durch die Strategien von Spieler $1$ indiziert, ihre Spalten durch die Strategien von Spieler $2$. Jedem Tupel $(x_1, x_2)$ aus je einer Strategie der beiden Spieler (also einem Strategieprofil) entspricht dann genau ein Eintrag der Matrix, welcher das Tupel $(c_1(x_1, x_2), c_2(x_1, x_2))$ ist.
\end{notation}

\begin{bsp}\label{bsp:SchereSteinPapier}
	Das Spiel Schere-Stein-Papier lässt sich wie folgt als strategisches Spiel beschreiben: Jeder der beiden Spieler hat eine dreielementige Strategiemenge, nämlich $X_1 = X_2 = \set{\text{\faHandScissorsO},\text{ \faHandRockO},\text{ \faHandPaperO}}$, sowie die durch die folgende Matrix gegebenen Kostenfunktion
	\begin{center}
		\begin{tabular}{c||c|c|c}
								& \faHandScissorsO 	& \faHandRockO	& \faHandPaperO 	\\\hline\hline
			\faHandScissorsO	& $(0,0)$				& $(1,{-1})$ 		& $({-1},1)$			\\\hline
			\faHandRockO		& $({-1},1)$			& $(0,0)$			& $(1,{-1})$			\\\hline
			\faHandPaperO		& $(1,{-1})$			& $({-1},1)$		& $(0,0)$
		\end{tabular}.
	\end{center}	
\end{bsp}

\begin{bsp}\label{bsp:GefangenenDilemma}
	Ein klassisches Beispiel der Spieltheorie ist das Gefangenendilemma (vgl. etwa \cite[Beispiel 3 (S. 65)]{Foundations}), in dem beide Spieler je zwei Strategien haben $X_1 = X_2 = \set{\text{gestehen},\text{ leugnen}}$ und die Kostenfunktionen durch folgende Matrix beschrieben werden:
	\begin{center}
		\begin{tabular}{c||c|c}
							& \text{gestehen} 	& \text{leugnen}	\\\hline\hline
			\text{gestehen}	& $(8,8)$			& $(0,10)$ 			\\\hline
			\text{leugnen}	& $(10,0)$			& $(1,1)$	
		\end{tabular}
	\end{center}	
\end{bsp}

\begin{notation}
	Zu einem Strategieprofil $x \in X$ und einer einzelnen Strategie $\hat{x}_i \in X_i$ bezeichnen wir einer Konvention aus \cite{Polyequilibrium} folgend mit $(x \mid \hat{x}_i)$ das Strategieprofil, das aus $x$ entsteht, wenn Spieler $i$ einseitig seine Strategie von $x_i$ auf $\hat{x}_i$ ändert.
\end{notation}

\begin{defn}
	Ein Strategieprofil $x \in X$ ist ein \emph{(reines) Nash-Gleichgewicht}, wenn für jeden Spieler $i \in I$ und jede seiner Strategien $\hat{x}_i$ gilt:
		\[c_i(x \mid \hat{x}_i) \geq c_i(x)\]
	Ein Nash-Gleichgewicht ist also ein Strategieprofil, aus dem heraus kein Spieler einen Anreiz für einen einseitigen Strategiewechsel hat.	
\end{defn}

\begin{bsp}
	Im Gefangenendilemma (\Cref{bsp:GefangenenDilemma}) gibt es genau ein Nash-Gleichgewicht, nämlich $(\text{gestehen},\text{ gestehen})$. Schere-Stein-Papier (\Cref{bsp:SchereSteinPapier}) hingegen besitzt kein Nash-Gleichgewicht.
\end{bsp}

In \cite[Definition 2.2]{KoordDummy} werden Koordinations- und Dummy-Spiele definiert. Daran anlehnend definieren wir hier zusätzlich skalierte Koordinationsspiele und Dummy-Spieler:
\begin{defn}
	Ein Spiel $\Gamma = (I, X, (c_i))$ heißt
	\begin{itemize}
		\item \emph{Koordinationsspiel}, wenn alle Spieler eine gemeinsame Kostenfunktion nutzen, wenn also gilt $c_i = c_\ia$ für alle Spieler $i,\ia \in I$.
		\item \emph{skaliertes Koordinationsspiel}, wenn jeder Spieler eine streng monoton skalierte Variante einer gemeinsamen Kostenfunktion verwendet, d.h. wenn es eine Funktion $c: X \to \IR$ und streng monotone\footnote{Mit streng monotonen Funktionen sind in dieser Arbeit immer ordnungserhaltende, also streng monoton \emph{wachsende} Funktionen gemeint.} Funktionen $f_i: \IR \to \IR$ gibt, sodass für jeden Spieler $c_i = f_i \circ c$ gilt.
	\end{itemize}
	Ein Spieler $i \in I$ heißt \emph{Dummy-Spieler}, wenn die alleinige Abweichung dieses Spielers von einem gegebenen Strategieprofil nie zu einer Änderung seiner Kosten führt. D.h. wenn für alle $x \in X, i \in I$ und $\hat{x}_i \in X_i$ gilt $c_i(x) = c_i(x \mid \hat{x}_i)$. 
	
	Das ganze Spiel $\Gamma$ heißt \emph{Dummy-Spiel}, wenn alle Spieler des Spiels Dummy-Spieler sind.
\end{defn}

\begin{beob}
	In einem (skalierten) Koordinationsspiel sind globale Minima der gemeinsamen Kostenfunktion Nash-Gleichgewichte des Spiels. Insbesondere haben damit endliche (skalierte) Koordinationsspiele immer mindestens ein Nash-Gleichgewicht.
	
	In einem Dummy-Spiel ist \emph{jedes} Strategieprofil ein Nash-Gleichgewicht, da ein einzelner Spieler durch einseitige Abweichung seine Kosten nie verändern und daher insbesondere nicht verringern kann.
\end{beob}


\subsection{Abweichungspfade}

Eine natürliche Methode ein Nash-Gleichgewicht in einem konkreten Spiel zu finden ist die folgende: Man beginnt bei einem beliebigen Strategieprofil und prüft, ob dieses bereits ein Nash-Gleichgewicht ist. Ist dies der Fall, ist man bereits fertig. Andernfalls gibt es wenigstens einen Spieler, der seine Kosten durch einen einseitigen Strategiewechsel verbessern kann. Dadurch erhält man ein neues Strategieprofil, für das man erneut prüft, ob es ein Nash-Gleichgewicht ist usw. Terminiert dieses Verfahren, so hat man ein Nash-Gleichgewicht gefunden. 

Im Laufe dieses Verfahrens erhält man eine Folge von Strategieprofilen, die jeweils durch Abweichung eines einzelnen Spielers auseinander hervorgehen, wobei dieser Spieler seine Kosten senkt. Eine solche Folge bezeichnen \citeauthor{MonShap} in \cite{MonShap} als \glqq Verbesserungspfad\grqq. Stellt man andere Bedingungen an die Kostenveränderung für den abweichenden Spieler, so erhält man analog weitere Arten von Pfaden im Strategieraum (vergleiche etwa \cite{CharExOrdPot} und \cite{BestRespPot}).

\begin{defn}\label{defn:Schritte}
	Ein Tupel von Strategieprofilen $(x^0, x^1) \in X\times X$ nennen wir einen \emph{Schritt}, wenn $x^1$ durch Abweichung eines einzelnen Spielers aus $x^0$ entsteht. Das heißt also, wenn es einen Spieler $i \in I$ gibt, sodass $x^1 = (x^0 \mid x^1_i)$ gilt. 
	
	Ein solcher Schritt heißt
	\begin{itemize}
		\item \emph{Nichtverschlechterungsschritt}, wenn sich der abweichende Spieler nicht verschlechtert, d.h. $c_i(x^0) \geq c_i(x^1)$.
		\item \emph{Verbesserungsschritt}, wenn sich der abweichende Spieler echt verbessert, wenn also $c_i(x^0) > c_i(x^1)$ gilt.
		\item \emph{Beste-Antwort-Schritt}, wenn der abweichende Spieler eine für ihn bestmögliche Antwort wählt, d.h. $\min_{\hat{x}_i \in X_i} c_i(x^0 \mid \hat{x}_i) = c_i(x^1)$.		
	\end{itemize}
\end{defn}

\begin{defn}\label{defn:Pfade}
	Eine Folge von Strategieprofilen $\gamma = (x^0, x^1, x^2, \dots) \in X^\IN$ ist ein \emph{Pfad}, wenn je zwei aufeinanderfolgende Strategieprofile einen Schritt bilden. Das heißt wenn es für jeden Schritt $k$ einen Spieler $i(k) \in I$ gibt, sodass das Profil $x^{k}$ aus $x^{k-1}$ durch alleinige Abweichung dieses Spielers entsteht, also $x^{k} = \left(x^{k-1} \mid x^{k}_{i(k)}\right)$.
	
	Ein solcher Pfad heißt
	\begin{itemize}
		\item \emph{Nichtverschlechterungspfad}, wenn jeder Schritt ein Nichtverschlechterungsschritt ist, d.h. für alle $k$ gilt $c_{i(k)}(x^{k-1}) \geq c_{i(k)}(x^{k})$.
		\item \emph{schwacher Verbesserungspfad}, wenn er ein Nichtverschlechterungspfad ist und mindestens ein Schritt ein Verbesserungsschritt ist.
		\item \emph{Verbesserungspfad}, wenn jeder Schritt ein Verbesserungsschritt ist.
		\item \emph{Beste-Antwort-Pfad}\footnote{In \cite{BestRespPot} werden solche Pfade als \glqq Beste-Antwort-kompatibel\grqq{} bezeichnet.}, wenn jeder Schritt ein Beste-Antwort-Schritt ist, wenn also gilt $\min_{\hat{x}_{i(k)} \in X_{i(k)}} c_{i(k)}(x^{k-1} \mid \hat{x}_i) = c_{i(k)}(x^k)$.
		\item \emph{schwacher Beste-Antwort-Verbesserungspfad}, wenn er ein schwacher Verbesserungspfad und ein Beste-Antwort-Pfad ist.
		\item \emph{Beste-Antwort-Verbesserungspfad}, wenn er ein Verbesserungspfad und ein Beste-Antwort-Pfad ist.
	\end{itemize}
\end{defn}

\begin{defn}\label{defn:Pfade2}	
	Für einen endlichen Pfad $\gamma = (x^0, \dots, x^n)$ ist
	\begin{itemize}
		\item die \emph{Gesamtänderung} entlang des Pfades $\PfadAend(\gamma)$ definiert als die Summe aller Änderungen für die jeweils abweichenden Spieler:
		\[\PfadAend(\gamma) \coloneqq \sum_{k=1}^{n} \left(c_{i(k)}(x^k) - c_{i(k)}(x^{k-1})\right) \]
		\item $\gamma$ ein \emph{abgeschlossener Verbesserungspfad}, wenn er ein Verbesserungspfad ist, der nicht mehr nach hinten verlängert werden kann, d.h. es keine Strategie $\hat{x}_i$ gibt, mit $c_{i}(x^n) > c_{i}(x^n \mid \hat{x}_i)$.
		\item $\gamma$ ein \emph{($n$-)Zykel}, wenn $x^0 = x^n$ und $n \geq 1$ gilt.
		\item $\invPf{\gamma} \coloneqq (x^n, x^{n-1}, \dots, x^0)$ der in die Gegenrichtung laufende Pfad.
	\end{itemize}

	Ist zusätzlich $\mu = (y^0, y^1, \dots)$ ein beliebiger bei $x^n$ beginnender Pfad (also $y^0 = x^n$), so ist $\gamma\Pfadkomp\mu \coloneqq (x^0, \dots, x^n, y^1, \dots)$ die Verknüpfung der beiden Pfade (und offensichtlich selbst wieder ein Pfad).
\end{defn}

\begin{beob}\label{beob:Pfade}
	Einige einfache Beobachtungen über Pfade:
	\begin{itemize}
		\item Ein endlicher Nichtverschlechterungspfad $\gamma$ ist genau dann ein schwacher Verbesserungspfad, wenn $\invPf{\gamma}$ kein Nichtverschlechterungspfad ist. 
		
		\item Zwei Pfade $\gamma$ und $\mu$ von $x$ nach $y$ haben genau dann die gleiche Gesamtänderung, wenn $\gamma\Pfadkomp\invPf{\mu}$ eine Gesamtänderung von $0$ hat. Es gilt also für zwei solche Pfade: 
		\[\PfadAend(\gamma) = \PfadAend(\mu) \iff \PfadAend(\gamma\Pfadkomp\invPf{\mu})=0 \]
		
		\item Jeder Beste-Antwort-Pfad ist automatisch ein Nichtverschlechterungspfad.
	\end{itemize}
\end{beob}

Der eingangs beschriebene natürliche Algorithmus zum Finden eines Nash-Gleichgewichtes terminiert genau dann immer, wenn alle Verbesserungspfade in dem gegebenen Spiel endlich sind. Diese Eigenschaft bezeichnen wir als als finite improvement property:

\begin{defn}
	Ein Spiel $\Gamma$ hat die \emph{finite improvement property (FIP)}, wenn jeder Verbesserungspfad endlich ist.
\end{defn}

\begin{beob}\label{beob:VerbPfadeundNGe}
	Das Ende eines abgeschlossenen Verbesserungspfades ist immer ein Nash-Gleichgewicht. Denn wäre dem nicht so, dann gäbe es wenigstens einen Spieler, der sich durch Abweichen noch verbessern kann - was zu einer Verlängerung des Verbesserungspfades führen würde. 
		
	Umgekehrt ist offenkundig auch jedes Nash-Gleichgewicht Ende wenigstens eines abgeschlossenen Verbesserungspfades (sogar Beste-Antwort-Verbesserungspfades) - nämlich des trivialen, nur aus diesem Strategieprofil bestehenden Verbesserungspfades.
\end{beob}

\begin{beob}\label{kor:ExVerbPfadExNG}
	Ein Spiel $\Gamma$ besitzt genau dann (mindestens) ein Nash-Gleichgewicht, wenn es (mindestens) einen abgeschlossenen (Beste-Antwort-)Verbesserungspfad besitzt. Ein Spiel mit FIP besitzt dementsprechend immer wenigstens ein Nash-Gleichgewicht.
\end{beob}

Alternativ können auch Beste-Antwort-Schritte dazu verwendet werden Nash-Gleichgewichte zu charakterisieren:

\begin{beob}\label{beob:BASchritteNG}
	Eine Strategieprofil $x \in X$ ist genau dann ein Nash-Gleichgewicht, wenn $(x,x)$ für jeden Spieler ein Beste-Antwort-Schritt ist.
\end{beob}

Beste-Antwort-Pfade treten bei sogenannten Beste-Antwort-Dynamiken auf: Bei diesen handelt es sich ebenfalls um Verfahren zum Auffinden von Nash-Gleichgewichten, im Unterschied zum zuvor beschriebenen Verfahren wechselt der abweichende Spieler hierbei allerdings nicht zu irgendeiner für ihn besseren Strategie, sondern wählt direkt die (momentan) beste der ihm zur Verfügung stehenden Strategien. 

Schließlich führen wir noch eigene Bezeichnungen für den Bereich eines Strategieraums ein, der von einem gegebenen Strategieprofil aus überhaupt mit Hilfe von Pfaden erreicht werden kann. 

\begin{defn}
	Eine Teilmenge $Y \subseteq X$ des Strategieraumes bezeichnen wir als \emph{Pfadzusammenhangskomponente}, wenn es für je zwei $x, y \in Y$ einen Pfad von $x$ nach $y$ in $Y$ gibt. Zu einem gegebeenen Strategieprofil $x \in X$ ist \emph{die Pfadzusammenhangskomponente von $x$} die (bezüglich Inklusion) maximale Pfadzusammenhangskomponente $Y_x \subseteq X$, welche $x$ enthält.
\end{defn}

\begin{beob}\label{beob:NPersSpieleNureinePfadzshkomp}
	Die Pfadzusammenhangskomponente eines Strategieprofils $x$ besteht aus allen Strategieprofilen, die sich an höchstens endlich vielen Stellen von $x$ unterscheiden, d.h.
	\[Y_x = \set{y \in X | \exists J \subseteq I: \abs{J} < \infty \text{ und } \forall i \in I\setminus J: x_i = y_i } \]
	Insbesondere ist also für $N$-Personenspiele der gesamte Strategieraum eine einzige Pfadzusammenhangskomponente.
\end{beob}


\subsection{Auslastungsspiele}

In diesem Abschnitt wollen wir noch eine besondere Klasse von Spielen kennenlernen, die erstmals von \citeauthor{RosenthalPotential} in \cite{RosenthalPotential} definierten Auslastungsspiele sowie einige Varianten davon. Verschiedene Eigenschaften dieser Spiele sowie insbesondere ihre Beziehungen zu Potentialspielen werden wir später in \Cref{sec:Auslastungsspiele} diskutieren.

\begin{defn}\label{def:Auslastungsmodel}
	Ein \emph{Auslastungsmodell $M$} ist ein Tupel $M = (I, R, S \coloneqq \prod_{i\in I} S_i, (g_r)_{r \in R})$. Dabei ist
	\begin{itemize}
		\item $I$ die Menge der Spieler,
		\item $R$ die Menge der zur Verfügung stehenden Ressourcen,
		\item $g_r: \IR_{\geq 0} \to \IR$ eine Funktion, welche die Kosten der Ressource $r$ in Abhängigkeit von ihrer Auslastung beschreibt, und
		\item $S_i \subseteq \PSet(R)$ eine (nicht-leere) Menge von Teilmengen der Ressourcenmenge, unter denen sich der Spieler $i$ für eine Teilmenge entscheiden kann. 
	\end{itemize}
\end{defn}

\begin{defn}\label{def:Auslastungsspiel}
	Ein Auslastungsmodell $M$ induziert durch die Lastfunktionen
		\[l_r: S \to \IN: s \mapsto \abs{\{i\in I \mid r \in s_i\}}\]
	und die Kostenfunktionen
		\[c_i: S \to \IR: s \mapsto \sum_{r \in R} g_r(l_r(s)) \]
	ein \emph{Auslastungsspiel} $\Gamma(M) := (I, S, (c_i)_{i \in I})$, wenn alle $c_i$ und alle $l_r$ auf ganz $S$ wohldefiniert sind (also die entsprechenden Mengen/Summen für jedes $s \in S$ endlich sind/absolut konvergieren).
\end{defn}

\begin{bem}\label{bem:AuslSpielWohldefiniertheit}
	Eine Lastfunktion $l_r$ ist genau dann wohldefiniert, wenn diese Ressource $r$ von höchstens endlich vielen verschiedenen Spielen benutzt werden kann. Das heißt wenn gilt: $\abs{\Set{i \in I | \exists s_i \in S_i: r \in s}} < \infty$. Sind alle $l_r$ bereits wohldefiniert, so ist eine hinreichende Bedingung für die Wohldefiniertheit von $c_i$, dass alle diesem Spieler zur Auswahl stehende Teilmengen von $R$ endlich sind. Diese Bedingung ist aber nicht notwendig.
	
	Insbesondere induzieren Auslastungsmodelle mit endlicher Ressourcen- und Spielerzahl daher immer ein Auslastungsspiel.
	
	Ist die Ressourcenmenge $R$ endlich, so ist das induzierte Auslastungsspiel ein endliches Spiel (d.h. dessen Strategieraum ist endlich). Ist umgekehrt der Strategieraum endlich und jede Strategie jedes Spielers enthält nur endlich viele Ressourcen, so können wir ohne Einschränkung davon ausgehen, dass auch $R$ endlich ist. Denn in diesem Fall tauchen nur endlich viele der Ressourcen in einer Strategie wenigstens eines der Spieler auf - und alle anderen Ressourcen können für die Betrachtung des Spiels ignoriert werden.
\end{bem}

\begin{bsp}
	Ein Beispiel für Auslastungsmodelle sind \emph{Netzwerkauslastungsmodelle}. Diese sind gegeben durch einen (nicht notwendigerweise endlichen) Multigraphen $G=(V,E)$, eine Kostenfunktion $g_e: \IR_{\geq 0} \to \IR$ für jede Kante $e \in E$ und für jeden Spieler $i \in I$ je ein ausgezeichneter Start- und Endknoten $s_i, t_i \in V$. Die Ressourcen sind dann die Kanten des Graphen, die Strategien von Spieler $i$ entsprechen den endlichen $s_i,t_i$-Pfaden in $G$. 
	
	Zusätzlich ist es möglich bestimmte Kanten nur für einen Teil der Spieler zuzulassen. Dadurch wird dann der Strategieraum der anderen Spieler entsprechend eingeschränkt.
\end{bsp}

Verallgemeinerungen von Auslastungsspielen erhalten wir durch das Einführen von Spielergewichten bzw. dem Abändern der Form, in der die Kosten berechnet werden:

\begin{defn}\label{def:gewAuslastungsspiel}
	Zusammen mit einem positiven Gewichtsvektor $(w_i)_{i\in I} \in \IR_{> 0}^I$ induziert ein Auslastungsmodell $M$ (wiederum unter der Voraussetzung, dass alle beteiligten Funktionen wohldefiniert sind)
	\begin{itemize}
		\item ein \emph{kostengewichtetes Auslastungsspiel} $\Gamma_c(M, w) := (I, S, (c_i)_{i \in I})$ durch die Kostenfunktion
		\[c_i: S \to \IR: s \mapsto \sum_{r \in R} w_i\cdot g_r(l_r(s)) \]
		und die gleiche Lastfunktion wie im ungewichteten Fall.
		\item ein \emph{lastgewichtetes Auslastungsspiel} $\Gamma_l(M, w) := (I, S, (c_i)_{i \in I})$ durch die Kostenfunktion wie im ungewichteten Fall und die Lastfunktion 
		\[l_r: S \to \IN: s \mapsto \sum\set{w_i | r \in s_i}.\]
		\item ein \emph{gewichtetes Auslastungsspiel} $\Gamma(M, w) := (I, S, (c_i)_{i \in I})$ durch die Kostenfunktion
		\[c_i: S \to \IR: s \mapsto \sum_{r \in R} w_i\cdot g_r(l_r(s)) \]
		und die Lastfunktion 
		\[l_r: S \to \IN: s \mapsto \sum\set{w_i | r \in s_i}.\]
	\end{itemize}	
\end{defn}

\begin{beob}
	Für (last-)gewichtete Auslastungsspiele ist es auch möglich, dass abzählbar unendlich viele Spieler eine Ressource nutzen können (und die entsprechende Lastfunktion dennoch wohldefiniert ist) - beispielsweise dann, wenn der Gewichtsvektor von der Form $(2^n)_{n \in \IN}$ ist.
\end{beob}

Einige weitere Varianten von Auslastungsspielen sind:

\begin{defn}\label{def:weitereAuslastungsspiel}
	\begin{itemize}
		\item Zusammen mit einer Familie streng monotoner Skalierungsfunktionen $(f_i: \IR \to \IR)_{i\in I}$ erhält man aus einem Auslastungsmodell ein \emph{skaliertes Auslastungsspiel} durch Verwenden der Kostenfunktionen
			\[c_i: S \to \IR: s \mapsto \sum_{r \in R} f_i(g_r(l_r(s))). \]
		\item Ein \emph{Auslastungsmodell/-spiel mit spielerspezifischen Ressourcenkosten} erhält man, wenn es zu jeder Ressource $r \in R$ und jedem Spieler $i \in I$ eine spielerspezifische Ressourcenkostenfunktion $g_r^i: \IR_{\geq 0} \to \IR$ gibt und die Kostenfunktionen der Spieler definiert sind als
			\[c_i: S \to \IR: s \mapsto \sum_{r \in R} g_r^i(l_r(s)).\]
	\end{itemize}
\end{defn}

\begin{bem}
	Jedes skalierte Auslastungsspiel ist insbesondere ein Auslastungsspiel mit spielerspezifischen Ressourcenkosten: Setze dazu $g_r^i \coloneqq f_i \circ g_r$.
\end{bem}

\begin{defn}
	Ein Auslastungsmodell/-spiel heißt \emph{nicht-anonym}, wenn die Kosten einer Ressource nicht nur von der Zahl bzw. dem Gesamtgewicht der sie nutzenden Spieler, sondern auch von deren Identität abhängt. Die Lastfunktionen sind dann also definiert als
		\[l_r: S \to \Pc(I): s \mapsto \Set{i \in I | r \in s_i }\]
	und die Ressourcenkostenfunktionen sind von der Form
		\[g_r: \Pc(I) \to \IR. \]
\end{defn}

Speziellere Klassen von Auslastungsspielen erhalten wir hingegen, wenn die Mengen der zulässigen Teilmengen der einzelnen Spieler zusätzliche Struktur besitzen. Der einfachste Fall hierfür ist, dass jeder Spieler immer nur genau eine Ressource auf einmal wählen kann:

\begin{defn}
	Ein Auslastungsmodell/-spiel heißt \emph{Singelton-Auslastungsmodell/-spiel}, wenn jede Strategie jedes Spielers aus genau einer Ressource besteht.
\end{defn}

Eine andere Möglichkeit ist, dass die zulässigen Teilmengen jedes Spielers Basen eines Matroids auf der Ressourcenmenge bilden:

\begin{defn}
	Ein ungewichtetes Auslastungsmodell/-spiel heißt \emph{Matroidmodell/-spiel}, wenn es für jeden Spieler $i \in I$ einen Matroid $M_i$ auf der Ressourcenmenge gibt, sodass $S_i$ gerade die Basen dieses Matroids sind.
\end{defn}

\begin{defn}
	Ein Teilmengensystem $M \subseteq \PSet(R)$ auf einer endlichen Menge $R$ heißt Matroid, wenn es die folgenden drei Axiome erfüllt (vgl. \cite[Abschnitt 39.1]{CombOptMatroide})
	\begin{enumerate}
		\item $\emptyset \in M$ (oder äquivalent dazu: $M \neq \emptyset$)
		\item $M$ ist abgeschlossen unter Inklusion, d.h. $\forall S \in M, T \subseteq R: T \subseteq S \implies T \in M$.
		\item $\forall S, T \in M: \abs{T} < \abs{S} \implies \exists r \in S \setminus T: T \cup \set{r} \in M$.
	\end{enumerate}
	Die Mengen in $M$ bezeichnen wir dann als \emph{unabhängige Mengen}, die bezüglich Inklusion maximalen Mengen unter diesen nennen wir \emph{Basen des Matroids}.
\end{defn}

Eine wichtige Eigenschaft, welche alle Matroide erfüllen, ist der starke Basisaustauschsatz (vgl. \cite[Theorem 29.12]{CombOptMatroide}):

\begin{satz}
	Seien $M$ ein Matroid, $B_1, B_2$ zwei Basen von $M$ und $q \in B_1 \setminus B_2$ beliebig. Dann gibt es ein Element $r \in B_2 \setminus B_1$, sodass das Vertauschen dieser beiden Elemente wieder zwei Basen ergibt, d.h. sowohl $B_1 \setminus \set{q} \cup \set{r}$ als auch $B_2 \setminus \set{r} \cup \set{q}$ sind ebenfalls Basen von $M$.
\end{satz}

Mittels Induktion ergibt sich daraus die Basen-Matching-Eigenschaft (\cite[Korollar 39.12a]{CombOptMatroide}):

\begin{kor}\label{kor:BasenMatchingEig}
	Seien $M$ ein Matroid und $B_1, B_2$ zwei Basen von $M$. Dann gibt es eine Partitionierung der symmetrischen Differenz $B_1 \symDiff B_2 \coloneqq B_1 \setminus B_2 \cup B_2 \setminus B_1$ in eine Menge von Tupeln der Form $(q,r)$ mit $q \in B_1\setminus B_2$ und $r \in B_2 \setminus B_1$, sodass $B_1 \setminus \set{q} \cup \set{r}$ ebenfalls eine Basis von $M$ ist. Es gibt also eine Menge von Tupeln $A \subseteq (B_1 \setminus B_2) \times (B_2 \setminus B_1)$, sodass gilt:
	\begin{itemize}
		\item $\forall q \in B_1 \setminus B_2: \exists! r \in B_2 \setminus B_1: (q,r) \in A$
		\item $\forall r \in B_2 \setminus B_1: \exists! q \in B_1 \setminus B_2: (q,r) \in A$
		\item $\forall (q,r) \in A: B_1 \setminus \set{q} \cup \set{r}$ ist eine Basis von $M$
	\end{itemize}
\end{kor}
