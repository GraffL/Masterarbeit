\section{Auslastungsspiele}\label{sec:Auslastungsspiele}

\subsection{Definitionen}

\begin{defn}\label{def:Auslastungsmodel}
	Ein \emph{Auslastungsmodell $M$} ist gegeben durch ein Tupel $(I, R, (S_i)_{i\in I}, K, (g_r)_{r \in R})$. Dabei ist
	\begin{itemize}
		\item $I$ die Menge der Spieler,
		\item $R$ die Menge der zur Verfügung stehenden Ressourcen,
		\item $S_i \subseteq \PSet(R)$ die Menge von endlichen Teilmengen der Ressourcenmenge, unter denen sich der Spieler $i$ für eine Teilmenge entscheiden kann,
		\item $K$, eine geordnete abelsche Gruppe, der Kostenraum der Ressourcen und
		\item $g_r: \IR \to K$ eine Funktion, welche die Kosten der Ressource $r \in R$ in Abhängigkeit von ihrer Auslastung beschreibt.
	\end{itemize}
\end{defn}

\begin{defn}\label{def:Auslastungsspiel}
	Jedes Auslastungsmodell $M$ induziert ein \emph{Auslastungsspiel} $\Gamma(M) := (I, S = (S_i)_{i\in I}, (K)_{i\in I}, (c_i)_{i \in I})$ durch die Kostenfunktion:
		\[c_i: S \to K: s \mapsto \sum_{r \in R} g_r(l_r(s)) \]
	wobei $l_r: S \to \IN: s \mapsto \abs{\{i\in I \mid r \in s_i\}}$ die \emph{Lastfunktion}\todo{Hier bekommt man wohl Probleme, wenn Spieler- bzw. Ressourcenmenge unendlich ist.}{} der Ressource ist.
\end{defn}

\begin{bsp}.
	
	\todo[inline]{Netzwerkauslastungsspiel}
\end{bsp}

\begin{defn}\label{def:gewAuslastungsspiel}
	Zusammen mit einem Gewichtsvektor $(w_i)_{i\in I}$ induziert ein Auslastungsmodell $M$ 
	\begin{itemize}
		\item ein \emph{kostengewichtetes Auslastungsspiel} $\Gamma_c(M, w) := (I, S, (K)_{i\in I}, (c_i)_{i \in I})$ durch die Kostenfunktion:
		\[c_i: S \to K: s \mapsto \sum_{r \in R} w_i\cdot g_r(l_r(s)) \]
		und die gleiche Lastfunktion wie im ungewichteten Fall.
		\item ein \emph{lastgewichtetes Auslastungsspiel} $\Gamma_l(M, w) := (I, S, (K)_{i\in I}, (c_i)_{i \in I})$ durch die Kostenfunktion wie im ungewichteten Fall und die Lastfunktion $l_r: S \to \IN: s \mapsto \sum\{w_i \mid r \in s_i\}$.
		\item ein \emph{gewichtetes Auslastungsspiel} $\Gamma(M, w) := (I, S, (K)_{i\in I}, (c_i)_{i \in I})$ durch die Kostenfunktion:
		\[c_i: S \to K: s \mapsto \sum_{r \in R} w_i\cdot g_r(l_r(x)) \]
		und die Lastfunktion $l_r: S \to \IN: s \mapsto \sum\{w_i \mid r \in s_i\}$.
	\end{itemize}	
\end{defn}

\begin{bsp}.
	
	\todo[inline]{gewichtetes Netzwerkauslastungsspiel}
\end{bsp}

\begin{defn}\label{def:skalierteAuslastungsspiel}
	\todo{Ist diese Definition wirklich sinnvoll?}
	Zusammen mit einer Skalierungsfunktionen $(f_i: K \to K_i)_{i\in I}$ und verallgemeinerten Lastfunktionen $(h_r: X \to \IR)_{r\in R}$ induziert ein Auslastungsmodell $M$ 
	\begin{itemize}
		\item ein \emph{kostenskaliertes Auslastungsspiel} $\Gamma_c(M, f_i) := (I, S, (K)_{i\in I}, (c_i)_{i \in I})$ durch die Kostenfunktion:
		\[c_i: S \to K: s \mapsto \sum_{r \in R} f_i(g_r(l_r(s))) \]
		und die gleiche Lastfunktion wie im ungewichteten Fall.
		\item ein \emph{lastskaliertes Auslastungsspiel} $\Gamma_l(M, h_r) := (I, S, (K)_{i\in I}, (c_i)_{i \in I})$ durch die Kostenfunktion:
		\[c_i: S \to K: s \mapsto \sum_{r \in R} g_r(h_r(s)) \]
		\item ein \emph{verallgemeinertes gewichtetes Auslastungsspiel} $\Gamma(M, f_i, h_r) := (I, S, (K)_{i\in I}, (c_i)_{i \in I})$ durch die Kostenfunktion:
		\[c_i: S \to K: s \mapsto \sum_{r \in R} f_i(g_r(h_r(s))).\]
	\end{itemize}	
\end{defn}


\subsection{Zusammenhänge zu anderen Spielen}

\todo[inline]{Sinnvollerweise eher nach Morphismen-Kapitel? Oder hier bereits als Motivation für dieses?}

\todo[inline]{Welche Endlichkeitsvoraussetzungen benötigt man hier jeweils?}

Berühmter Satz von \cite{MonShap}

\begin{satz}
	Jedes Auslastungsspiel besitzt ein exaktes Potential und jedes exakte Potentialspiel ist äquivalent zu einem Auslastungsspiel.
\end{satz}

Für gewichtete Auslastungsspiele zeigen \citeauthor{CharExGewPotinWCG} in \cite{CharExGewPotinWCG}:

\begin{satz}
	Gegeben eine Menge von stetigen Funktionen $C$. Dann besitzt genau dann jedes gewichtete Auslastungsspiel, welches nur Funktionen aus $C$ als Kostenfunktionen verwendet, ein exaktes Potential, wenn $C$ ausschließlich affin lineare Funktionen enthält.
	
	Und genau dann besitzt jedes gewichtete Auslastungsspiel, welches nur Funktionen aus $C$ als Kostenfunktionen verwendet, ein gewichtetes Potential, wenn $C$ ausschließlich affin lineare Funktionen oder ausschließlich Funktionen der Form $c(l) = a_c\cdot b^l + d_c$ enthält.
\end{satz}

Für allgemeine Spiele gilt nach \cite{ReprOfFiniteGamesAsNCG}

\begin{satz}
	Jedes Spiel ist äquivalent zu einem gewichteten Auslastungsspiel (sogar Netzwerkauslastungsspiel mit ...).
\end{satz}