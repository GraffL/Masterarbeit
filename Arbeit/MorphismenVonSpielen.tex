\section[Morphismen]{Morphismen und Isomorphismen von Spielen}\label{sec:Morphismen}

\todo[inline]{Motivation: Warum betrachtet man überhaupt Morphismen von Spielen. Wie induzieren diese Isomorphismen?}

Mögliche Motivationen:
\begin{itemize}
	\item Jedes Potential definiert selbst wieder ein Spiel mit einer gemeinsamen Kostenfunktion für alle Spieler. Und in einem gewissen Sinne ist dieses Spiel \glqq äquivalent\grqq{} zum ursprünglichen Spiel (z.B. gleiche Gleichgewichtspunkte). Das Hin- und Herwechseln zwischen diesen beiden Versionen eines Spiels kann man durch Morphismen beschreiben und die Äquivalenz der beiden wird dann dadurch sichtbar, dass diese Morphismen \emph{Isomorphismen} sind.
	\item \todo[inline]{Umgekehrt kann man aus Isomorphiebegriffen neue Potentialbegriffe definieren (vgl. \cite{BestRespEq})}
	\item Die Äquivalenz zwischen exakten Potentialspielen und Auslastungsspielen wird durch Morphismen beschrieben.
	\item Kategorientheoretische Sicht: Um eine Kategorie (hier: die der Spiele) zu verstehen, muss man ihre Morphismen kennen. Kennt man diese, so ergeben sich aus diesen auf natürliche Weise weitere Begriffe wie Isomorphismen von Spielen, Summen oder Produkte von Spielen.
\end{itemize}

\todo[inline]{Diese Punkte (insbesondere den letzten) näher ausführen?}

\subsection{Definitionen}

Zu zwei gegebenen Spielen $\Gamma = (I, X, (c_i)_{i\in I})$ und $\Delta = (J, Y, (d_j)_{j\in J})$ kann man wie folgt eine Abbildung $(\sigma, \phi): \Gamma \to \Delta$ zwischen diesen beiden definieren:
\begin{itemize}
	\item Eine Abbildung $\sigma: J \to I$ zwischen den Spielermengen und
	\item für jeden Spieler $j \in J$ eine Abbildung $\phi_j: X_{\sigma(j)} \to Y_j$ seiner Strategien.
\end{itemize}

Diese Art und Weise Abbildungen zwischen zwei Spielen zu definieren ergibt sich aus dem - noch deutlich allgemeineren - Ansatz hierzu von \citeauthor{LapGameCat} in \cite{LapGameCat}\footnote{\citeauthor{LapGameCat} lässt in seiner Definition allerdings alle beteiligten Abbildungen in die jeweils umgekehrte Richtung gehen. Wir betrachten hier also gewissermaßen die zur dort definierten duale Kategorie.}. Hierbei orientieren wir uns vor allem an dem dort in Kapitel 4 vorgestellten Morphismusbegriff für topologische Spiele (in denen die Spielermenge ein topologischer Raum und der Strategieraum eine Garbe über diesem ist) bzw. dessen Spezialisierung für Spiele mit diskreter Spielermenge aus Kapitel 5. 

In diesem Kontext ergibt es sich etwas natürlicher, dass die Abbildung zwischen den Spielermengen in die entgegengesetzte Richtung zu der zwischen den Strategieräumen verläuft. Eine andere, auch im hier vorliegenden Kontext sinnvolle Begründung hierfür liefert aber die folgende Bemerkung:

\begin{bem}
	Alle Strategieabbildungen $\phi_j$ zusammen induzieren eine \emph{Strategieprofilabbildung}
	\[\phi: X \to Y: x=(x_i)_{i\in I} \mapsto \phi(x) := \left(\phi_j(x_{\sigma(j)})\right)_{j \in J} \]
\end{bem}

Würde die Abbildung $\sigma$ zwischen den Spielermengen in die andere, \glqq natürlichere\grqq{} Richtung verlaufen (also von $I$ nach $J$), so müssten wir zusätzlich fordern, dass diese bijektiv ist, damit in der obigen Form Strategieprofile auf Strategieprofile abgebildet werden (vgl. etwa \cite{CatGameTheory}). Denn nur dann wäre $\phi(x) \coloneqq \left(\phi_i(x_i)\right)$ wieder ein vollständiges Strategieprofil in $Y$.

\begin{bem}\todo{Genauer beschreiben...}
	Möchte man trotzdem Spieler- und Strategieabbildungen in die gleiche Richtung haben, so kann man einem alternativen Ansatz zur Definition von Morphismen von Spielen folgen, welcher von \citeauthor{Foundations} in \cite{Foundations} verwendet wird. Darin werden einzelne Strategien nicht zwangsläufig wieder auf einzelne Strategien abgebildet, sondern können gleich auf ganze Teilmengen des Bildstrategieraums abgebildet werden. Von diesem Morphismentyp werden dann verschiedene \glqq approximativ kostenerhaltende\grqq{} Varianten betrachtet und die sich dadurch ergebende Kategorie studiert.
\end{bem}

\begin{notation}
	In den meisten Fällen wird die Abbildung zwischen den Spielermengen bijektiv sein. In diesen Fällen werden wir zur Vereinfachung der Notation ohne Einschränkung davon ausgehen, dass die Spielermengen beider an der Abbildung beteiligten Spiele bereits gleich und geeignet permutiert sind, sodass $\sigma$ die Identitätsabbildung ist. Damit kann diese in der Notation weggelassen werden und die Abbildung zwischen den Spielen besteht nur noch aus den Abbildungen $\phi_i: X_i \to Y_i$ zwischen den Strategieräumen.
\end{notation}

Abbildungen der obigen Form nehmen noch keinerlei Rücksicht auf die Kostenfunktionen der jeweiligen Spiele. Da diese aber in der Regel die interessierenden Eigenschaften eines Spiels (wie beispielsweise Gleichgewichte) festlegen, werden derartige Abbildungen im Allgemeinen noch wenig Aussagen über die beteiligten Spiele ermöglichen. So besagt der durch diese Art Abbildungen induzierte Isomorphiebegriff bspw. nur, dass zwei Spiele Spieler- und Strategiemengen gleicher Kardinalität besitzen.

Echte Morphismen zwischen Spielen sollten folglich noch mehr der Struktur eines Spiels erhalten, insbesondere in irgendeiner Form \glqq verträglich\grqq{} mit den Kostenfunktionen sein. Je nach dem, welche Eigenschaften die Morphismen (und insbesondere die dadurch induzierten Isomorphismen) erhalten sollen, erhält man so unterschiedlich starke Verträglichkeitsbedingungen. Einige Möglichkeiten dafür werden wir nun kennenlernen.

Eine relative starke Forderung ist die, dass Morphismen \emph{kostenerhaltend} sein müssen, wie sie in \cite{ReprOfFiniteGamesAsNCG} gestellt wird:

\begin{defn}
	Ein Morphismus $(\sigma, \phi_j)$ von $\Gamma = (I, X, (c_i)_{i\in I})$ nach $\Delta = (J, Y, (d_j)_{j\in J})$ heißt \emph{kostenerhaltend}, wenn für alle Strategieprofile $x \in X$ und jeden Spieler $j \in J$ gilt:
		\[c_{\sigma(j)}(x) = d_j(\phi(x)) \]
	Ist ein solcher Morphismus gleichzeitig ein Isomorphismus, so nennen wir $\Gamma$ und $\Delta$ \emph{äquivalent}.
\end{defn}

\begin{bem}
	Zwei Spiele sind also genau dann äquivalent, wenn sie sich ausschließlich durch Umbenennung der Strategien ineinander überführen lassen. In \cite{MonShap} (S. 133) wird dies als Isomorphie von Spielen bezeichnet.
\end{bem}

\begin{bsp}\label{bsp:Koordinationsspiel}
	Ein Spiel $\Delta = (J, Y, (d_j)_{j \in J})$ ist genau dann ein Koordinationsspiel, wenn es einen kostenerhaltenden Morphismus mit surjektiver Strategieprofilabbildung von einem 1-Personenspiel nach $\Delta$ gibt.
	
	Ist nämlich $\Delta$ ein Koordinationsspiel, so definiert man ein 1-Personenspiel $K \coloneqq (\{\ast\}, X, (c_\ast))$ mit $X = X_\ast \coloneqq Y$ und $c_\ast(y) \coloneqq d_{\hat{\jmath}}(y)$ für einen beliebigen, aber festen Spieler $\hat{\jmath} \in J$. Ferner definiert man den folgenden Morphismus $(\sigma, \phi): K \to \Delta$:
	\begin{align*}
		\sigma:	&&J		\to	 \{\ast\}:	&&j	\mapsto	\ast  \\
		\phi_j:	&&X_\ast	\to	 Y_j:	&&y	\mapsto	y_j
	\end{align*}	
	Dieser ist kostenerhaltend, denn für jedes Strategieprofil $y \in X = Y$ und jeden Spieler $j \in J$ gilt:
	\[c_{\sigma(j)}(y) = c_\ast(y) = d_{\hat{\jmath}}(y) \overset{\Gamma \text{ Koord.spiel}}{=} d_j(y) = d_j(\phi(y))\]
	Zudem ist offenbar die Strategieprofilabbildung $\phi$ surjektiv.
	
	Sind umgekehrt ein 1-Personenspiel $K = (\{\ast\}, X, (c_\ast))$ sowie ein kostenerhaltender Morphismus  $(\sigma, \phi): K \to \Delta$ mit surjektiven Strategieabbildungen gegeben, dann ist $\Delta$ bereits ein Koordinationsspiel. Denn aufgrund der Surjektivität von $\phi$ gibt es zu jedem Strategieprofil $y \in Y$ ein Strategieprofil $x \in X$ mit $\phi(x) = y$. Da $(\sigma, \phi)$ außerdem kostenerhaltend ist, folgt dann für je zwei Spieler $j, \hat{\jmath} \in J$: 
		\[d_{\hat{\jmath}}(y) = d_{\hat{\jmath}}(\phi(x)) = c_{\sigma(\hat{\jmath})}(x) = c_\ast(x) = c_{\sigma(j)}(x) = d_j(\phi(x)) = d_j(y)\]
	Also ist $\Gamma$ tatsächlich ein Koordinationsspiel.
\end{bsp}

Dieses Beispiel formalisiert die Intuition, dass in einem Koordinationsspiel alle Spieler ein gemeinsames Ziel haben und daher zusammen \glqq wie ein Spieler\grqq{} spielen (d.h. das Koordinationsspiel kann durch ein 1-Personenspiel simuliert werden).\todo{Stimmt das so?}

\begin{bsp}.
	
	\todo[inline]{Kann man auch Dummy-Spiele auf derartige Weise beschreiben?}
\end{bsp}

\begin{bsp}
	Gegeben zwei Spiele $\Gamma = (I, X, (c_i)_{i\in I})$ und $\Delta = (J, Y, (d_j)_{j\in J})$, so können wir daraus ein neues Spiel konstruieren: Das Produkt bezüglich kostenerhaltenden Morphismen: $\Gamma \times \Delta \coloneqq (K \coloneqq I \dot{\cup} J, X\times Y, (e_k)_{k \in K})$. Dabei ist $e_k(x,y) \coloneqq c_k(x)$ für $k \in K$ und analog für $k \in J$. \todo{Gibt es irgendeine anschauliche Konstruktion von Spielen, die sich als Produkt beschreiben lässt? Falls nicht, eher weglassen.}
\end{bsp}


In \cite{LapGameCat} stellt \citeauthor{LapGameCat} folgende schwächere Verträglichkeitsbedingung:

\begin{defn}
	Ein Morphismus $(\sigma, \phi_j)$ von $\Gamma = (I, X, (c_i)_{i\in I})$ nach $\Delta = (J, Y, (d_j)_{j\in J})$ heißt \emph{monoton}, wenn es für jeden Spieler $j \in J$ eine monotone \emph{Kostenabbildung} $f_j: \IR \to \IR$ gibt, sodass das folgende Diagramm kommutiert:	
	\begin{center}
		\begin{tikzcd}
			X \arrow[r, "\phi"] \arrow[d, "c_{\sigma(j)}"] 	& Y \arrow[d, "d_j"] \\
			\IR \arrow[r, "f_j"]							& \IR
		\end{tikzcd}
	\end{center}
\end{defn}

\begin{bem}
	Gibt es eine derartige Abbildung $f_j$ für einen Spieler $j \in J$, so gilt für je zwei Strategieprofile $x, \hat{x} \in X$:
		\begin{align}\label{eq:CharExMonMor}
			c_{\sigma(j)}(x) \leq c_{\sigma(j)}(\hat{x}) \implies d_j(\phi(x)) \leq d_j(\phi(\hat{x}))
		\end{align}
	Ist umgekehrt diese Bedingung für je zwei Strategieprofile eines Spielers $j \in J$ erfüllt und zudem $d_j(\phi(X)) \subseteq \IR$ beschränkt oder $c_{\sigma(j)}(X) \subseteq \IR$ in beide Richtungen unbeschränkt, so gibt es eine solche Abbildung $f_j$.
\end{bem}

\begin{proof}
	Existiert eine Abbildung $f_j$, so gilt für zwei Strategieprofile $x, \hat{x} \in X$ mit $c_{\sigma(j)}(x) \leq c_{\sigma(j)}(\hat{x})$:
		\[d_j(\phi(x)) = f_j\circ c_{\sigma(j)}(x) \leq f_j\circ c_{\sigma(j)}(\hat{x}) = d_j(\phi(\hat{x}))\]
		
	Ist $d_j(\phi(X)) \subseteq \IR$ beschränkt oder $c_{\sigma(j)}(X) \subseteq \IR$ in beide Richtungen unbeschränkt, so können wir folgende Abbildung $f_j: \IR \to \IR$ definieren:
		\[f_j: \IR \to \IR: t \mapsto \begin{cases}
			\sup \Set{d_j(\phi(x)) | x \in X},								&\not\exists x \in X: c_{\sigma(j)}(x) \geq t \\
			\inf\, \Set{d_j(\phi(x)) | x \in X, c_{\sigma(j)}(x) \geq t}, 	&t \geq 0 \text{ und } \exists x \in X: c_{\sigma(j)}(x) \geq t\\
			\sup \Set{d_j(\phi(x)) | x \in X, c_{\sigma(j)}(x) \leq t}, 	&t <    0 \text{ und } \exists x \in X: c_{\sigma(j)}(x) \leq t\\
			\inf\,\, \Set{d_j(\phi(x)) | x \in X},							&\not\exists x \in X: c_{\sigma(j)}(x) \leq t 
		\end{cases} \]
	Gilt Eigenschaft \eqref{eq:CharExMonMor}, so ist diese Abbildung monoton und es gilt $f_j\circ c_{\sigma(j)} = d_j \circ \phi$, denn für ein festes Strategieprofil $\hat{x} \in X$ mit $c_{\sigma(j)}(\hat{x}) \geq 0$ gilt dann:
		\[\forall x \in X: c_{\sigma(j)}(x) \geq c_{\sigma(j)}(\hat{x}) \overset{\eqref{eq:CharExMonMor}}{\implies} d_j(x) \geq d_j(\hat{x}) \]
	Also nach Definition von $f_j$:
		\[f_j \circ c_{\sigma(j)}(\hat{x}) = f_j (c_{\sigma(j)}(\hat{x})) = \inf\Set{d_j(\phi(x)) | x \in X, c_{\sigma(j)}(x) \geq c_{\sigma(j)}(\hat{x}) }= d_j(\phi(\hat{x})) = d_j\circ \phi(\hat{x})\]
	Analog folgt dies für Strategieprofile $\hat{x} \in X$ mit $c_{\sigma(j)}(\hat{x}) < 0$.
\end{proof}

\begin{prop}\label{prop:MonIsoKoordDannskPot}
	Ist ein Spiel monoton isomorph zu einem Koordinationsspiel, so besitzt es ein skaliertes Potential.
\end{prop}

\begin{proof}
	Sei also $(\sigma, \phi): \Gamma =(I, X, (c_i)) \to K = (J, Y, (d))$ ein solcher monotoner Isomorphismus. Dann gibt es folglich monotone Abbildungen $f_j$ und $g_i$, sodass alle Diagramme der folgenden Form kommutieren:
	
	\begin{center}
			\begin{tikzcd}
			X \arrow[r, "\phi"] \arrow[d, "c_i"] \arrow[rr, bend left=60, "\id"]	& Y \arrow[r, "\phi^{-1}"] \arrow[d, "d(i)"] 	& X \arrow[d, "c_i"] \\
			\IR \arrow[r, "f_{\sigma^{-1}(i)}"'] \arrow[rr, bend right=60, "\id"']	& \IR \arrow[r, "g_i"']										& \IR
		\end{tikzcd}
	\end{center}

	Dann ist $P: X \to \IR: x \mapsto d \circ \phi(x)$ ein skaliertes Potential mit Skalierungsfunktionen $g_i$, denn es gilt allgemein
		\[g_i \circ d \circ \phi = c_i \circ \phi^{-1} \circ \phi = c_i \circ \id = c_i \]
	und damit im Besonderen
		\[g_i \circ d \circ \phi(x) - g_i \circ d \circ \phi(x \mid \hat{x}_i) = c_i(x) - c_i(x \mid \hat{x}_i) .\]
	Ferner sind die $g_i$ nach Voraussetzung monoton und wegen $f_{\sigma^{-1}(i)} \circ g_i = \id$ injektiv, also insgesamt streng monoton.
\end{proof}

\begin{kor}
	Sei $(\sigma, \phi): K \to \Gamma$ ein monotoner Morphismus mit surjektiver Strategieprofilabbildung und bijektiven Kostenabbildungen von einem $1$-Personenspiel $K$ nach $\Gamma$. Dann besitzt $\Gamma$ ein skaliertes Potential.
\end{kor}

\begin{proof}
	Wir konstruieren ein zu $\Gamma$ monoton isomorphes Spiel $\Delta$ und einen kostenerhaltenden Morphismus mit surjektiver Strategieprofilabbildung von $K$ nach $\Delta$. Mit \Cref{bsp:Koordinationsspiel} ist $\Delta$ dann ein Koordinationsspiel und nach \Cref{prop:MonIsoKoordDannskPot} besitzt $\Gamma$ folglich ein skaliertes Potential.
	
	\todo[inline]{Diagramm zur Veranschaulichung des Beweises. Wie beschriftet man die Pfeile vernünftig?}	
	
	Seien also $\Gamma = (I, X, (c_i))$  und $K = (\set{\ast}, Z_\ast, (e_\ast))$ die gegebenen Spiele und $f_i: \IR \to \IR$ die (bijektiven) Kostenabbildungen zu $(\sigma, \phi)$. Dann definieren wir $\Delta \coloneqq (I, X, (f_i^{-1}\circ c_i)_{i \in I})$. Nun hat $(\sigma, \phi): K \to \Delta$ nach Voraussetzung eine surjektive Strategieprofilabbildung und ist ferner kostenerhaltend, denn für jedes $z \in Z_\ast$ gilt:
		\[f_i^{-1}\circ c_i \circ \phi(z) = f_i^{-1} \circ f_i \circ e_\ast(z) = e_\ast(z) = e_{\sigma(i)}(z) \]
	Weiter sind $(\id, \id): \Delta \to \Gamma$ und die Umkehrabbildung $(\id, \id): \Gamma \to \Delta$ monoton, denn
	\begin{center}
		\begin{tikzcd}
			X \arrow[r, "\id"] \arrow[d, "f_i^{-1}\circ c_i"'] 	& X \arrow[d, "c_i"] \\
			\IR \arrow[r, "f_i"]								& \IR
		\end{tikzcd}
			und
		\begin{tikzcd}
			X \arrow[d, "f_i^{-1}\circ c_i"'] 	& X \arrow[l, "\id"] \arrow[d, "c_i"] \\
			\IR 								& \IR \arrow[l, "f_i^{-1}"]
		\end{tikzcd}
	\end{center}		
	kommutieren offenbar. Also sind $\Delta$ und $\Gamma$ tatsächlich monoton isomorph.\todo{Ist dieser Satz wirklich sinnvoll? Insbesondere mit diesem Beweis?}
\end{proof}

Dies ist also bereits ein Beispiel dafür, wie man die Existenz eines Potentials durch die Existenz eines passenden Isomorphismus zu einem Koordinationsspiel ausdrücken kann. Allerdings ist die Verträglichkeitsbedinung an monotone Morphismen noch zu stark um auf diesem Weg auch eine notwendige Bedingung für die Existenz eines skalierten zu erhalten. Dies liegt daran, dass hier eine globale Montonie gefordert wird, während die meisten Potentiale lediglich lokale Monotonie benötigen.

\todo[inline]{Beispiel für die Nicht-Notwendigkeit finden!}

Wir schränken nun also die Verträglichkeitsbedingung auf Nachbarschaften ein und kommen so zum Begriff eines ordinalen Morphismus. Einen entsprechenden (Iso-)Morphismusbegriff für exakte Potentiale definiert \citeauthor{ReprOfFiniteGamesAsNCG} in \cite{ReprOfFiniteGamesAsNCG}. Und analog hierzu lassen sich auch für die anderen Potentialbegriffe passende Begriffe eines Morphismus (und damit eines Isomorphismus) definieren:

\begin{defn}\label{def:PotentialMorphismen}
	Ein Morphismus $(\sigma, \phi)$ von $\Gamma = (I, X, (c_i)_{i\in I})$ nach $\Delta = (J, Y, (d_j)_{j\in J})$ heißt
	\begin{itemize}
		\item \emph{exakt}, wenn für alle Strategieprofile $x \in X$, Spieler $j \in J$ und Strategien $\hat{x}_{\sigma(j)}$ gilt:
		\[c_{\sigma(j)}(x) - c_{\sigma(j)}(x \mid \hat{x}_{\sigma(j)}) = d_j(\phi(x)) - d_j(\phi(x \mid \hat{x}_{\sigma(j)}))\]
		\item \emph{gewichtend}, wenn es einen Vektor $(w_j)_{j \in J} \in \IR_{\geq 0}^J$ gibt, sodass für alle Strategieprofile $x \in X$, Spieler $j \in J$ und Strategien $\hat{x}_{\sigma(j)}$ gilt:
		\[c_{\sigma(j)}(x) - c_{\sigma(j)}(x \mid \hat{x}_{\sigma(j)}) = w_j\cdot\left(d_j(\phi(x)) - d_j(\phi(x \mid \hat{x}_{\sigma(j)}))\right)\]
		\item \emph{skalierend}, wenn es streng monotone Funktionen $f_j$ gibt, sodass für alle Strategieprofile $x \in X$, Spieler $j \in J$ und Strategien $\hat{x}_{\sigma(j)}$ gilt:
		\[c_{\sigma(j)}(x) - c_{\sigma(j)}(x \mid \hat{x}_{\sigma(j)}) = f_j(d_j(\phi(x))) - f_j(d_j(\phi(x \mid \hat{x}_{\sigma(j)})))\]
		\item \emph{ordinal}, wenn für alle Strategieprofile $x \in X$, Spieler $j \in J$ und Strategien $\hat{x}_{\sigma(j)}$ gilt:
		\[c_{\sigma(j)}(x) < c_{\sigma(j)}(x \mid \hat{x}_{\sigma(j)}) \implies d_j(\phi(x)) < d_j(\phi(x \mid \hat{x}_{\sigma(j)}))\]
		\item \emph{beste Antwort-erhaltend}, wenn für alle Spieler $j \in J$ und Strategieprofile $x \in X$ gilt:
		\[\phi(\arg \min_{x_{\sigma(j)} \in X_{\sigma(j)}}c_{\sigma(j)}(x \mid x_{\sigma(j)})) \subseteq \arg \min_{y_j \in Y_j} d_j(\phi(x) \mid y_j)\]
	\end{itemize}	
\end{defn}

Während die Verknüpfung zweier kostenerhaltender Morphismen wieder einen kostenerhaltenden Morphismus gibt und die zweier monotoner wieder einen monotonen, ist dies bei exakten/gewichteten/\dots im Allgemeinen nicht der Fall.  

\begin{bsp}
	Betrachte die folgenden drei 1- bzw. 2-Personenspiele:\todo{Als Auszahlungsmatrizen darstellen?}
	\begin{itemize}
		\item $\Gamma \coloneqq (\set{1}, \set{t, b}, c_1)$, wobei die Kostenfunktion $c_1$ wie folgt definiert ist: $c(t) = 0, c(b) = 1$.
		\item $\Delta \coloneqq (\set{1, 2}, \set{t, b} \times \set{l, r}, (d_1, d_2))$ mit Kostenfunktionen definiert durch folgende Auszahlungsmatrix (mit Einträgen $(d_1(\_), d_2(\_))$:
		\begin{center}
				\begin{tabular}{c||c|c}
					& l 		& r 		\\\hline\hline
				t	& $(0,0)$	& $(0,1)$	\\\hline
				b	& $(1,0)$	& $(1,1)$ 
			\end{tabular}
		\end{center}
		\item $E \coloneqq (\set{1, 2}, \set{t, b} \times \set{l, r}, (e_1, e_2))$ mit Kostenfunktionen definiert durch folgende Auszahlungsmatrix (mit Einträgen $(e_1(\_), e_2(\_))$:
		\begin{center}
			\begin{tabular}{c||c|c}
				& l 		& r 		\\\hline\hline
			t	& $(1,1)$	& $(0,1)$	\\\hline
			b	& $(1,0)$	& $(0,0)$ 
			\end{tabular}
		\end{center}
	\end{itemize}
	Dann sind sowohl die Abbildung $(\sigma, \phi): \Gamma \to \Delta$ mit $\sigma(1) = \sigma(2) = 1$ und $\phi_1(t) = t, \phi_1(b) = b, \phi_2(t) = l, \phi_2(b) = r$ als auch die Abbildung $(\tau, \psi): \Delta \to E$ mit $\tau = \id$ und $\psi_i = \id$ exakt, wie man durch einfaches Nachprüfen\todo{Nachprüfen!!!} der jeweiligen Bedingungen sieht. Gleichzeitig ist aber die Verknüpfung der beiden Abbildungen $(\tau, \psi)\circ(\sigma, \phi) = (\sigma\circ\tau, \psi\circ\phi)$ nicht einmal ordinal, denn es gilt
		\[c_{\sigma\circ\tau(1)}(t) = c_1(t) = 0 < 1 = c_1(b) = c_{\sigma\circ\tau(1)}(t) = c_1(t),\]
	aber gleichzeitig
		\[e_1(\psi\circ\phi(t)) = e_1(t,l) = 1 > 0 = e_1(b,r) = e_1(\psi\circ\phi(b)).\]
\end{bsp}

Insbesondere erhält man also keine Kategorie von Spielen im Sinne der Kategorientheorie, wenn man sich nur auf exakte/gewichtete/\dots Morphismen beschränkt (was für kostenerhaltende/monotone Morphismen der Fall ist und für monotone Morphismen in \cite{LapGameCat} auch weiter untersucht wird\todo{bessere Formulierung}). Man kann den Erhalt von Exaktheit/Gewichtetheit/\dots allerdings wieder sicherstellen, wenn man sich noch weiter auf Morphismen mit injektiver Spielerabbildung einschränkt:

\begin{lemma}
	Die Verknüpfung zweier exakter/gewichteter/skalierter/ordinaler/Bester-Antwort-Morphismen mit injektiven Spielerabbildungen ist wieder ein solcher Morphismus.
\end{lemma}

\begin{proof}
	Die Aussage des Lemmas ergibt sich aus folgender Beobachtung:\todo{Weiter...}
\end{proof}
	
\todo[inline]{einer der beiden/beide bijektiv auf ... sind?}

Es ist nun leicht zu sehen, dass diese Isomorphismusbegriffe tatsächlich dazu definiert werden können, die Existenz der verschiedenen Potentiale zu definieren:

\begin{bem}
	Ein Spiel $\Delta$ besitzt genau dann ein exaktes/gewichtetes/ordinales/Beste-Antwort-Potential, wenn es exakt/gewichtet/ordinal/beste Antwort-isomorph zu einem Koordinationsspiel ist.
\end{bem}\todo{Beweis?}

\todo[inline]{Zu beste Antwort: vgl. \cite{FictPlayProp}, \cite{BestRespEq}}

\begin{beob}
	Es gelten folgende Beziehungen zwischen den verschiedenen Morphismenbegriffen:
	\begin{center}
		\begin{tikzpicture}
		\node[] (ke) {kostenerhaltend};
		\node[right of=ke, node distance=7em] (ex) {exakt};
		\node[right of=ex, node distance=7em] (gew) {gewichtet};
		\node[right of=gew, node distance=7em] (sk) {skaliert};
		\node[right of=sk, node distance=7em] (ord) {ordinal};
		
		\node[above of=gew, node distance=2em] (mon) {monoton};
		
		\node[below of=ord, node distance=2em] (BA) {Beste Antwort};
		
		\draw[-implies, double equal sign distance] (ke) -- (ex);
		\draw[-implies, double equal sign distance] (ex) -- (gew);
		\draw[-implies, double equal sign distance] (gew) -- (sk);
		\draw[-implies, double equal sign distance] (sk) -- (ord);
		\draw[-implies, double equal sign distance] (ke) -- (mon.west);
		\draw[-implies, double equal sign distance] (sk) -- (BA.west);
		\end{tikzpicture}
	\end{center}
\end{beob}\todo{Wie sieht es mit Isomorphismen aus? Speziell auf Zusammenhang von skaliert und monoton eingehen!}


\subsection{Beweise der Potentialsätze}

\todo[inline]{Ordinale Morphismen erhalten Verbesserungen und reflektieren Nicht-Verschlechterung:}

\begin{prop}\label{prop:NVReflVerbErh}
	Sei $(\sigma, \phi): \Gamma \to \Delta$ ein ordinaler Morphismus mit surjektiver Spielerabbildung $\sigma$ und $\gamma = (x^0, x^1, \dots)$ ein Pfad in $X$, sodass auch $\phi(\gamma) \coloneqq (\phi(x^0), \phi(x^1), \dots)$ ein Pfad in $Y$ ist. Ist dann $\phi(\gamma)$ ein Nicht-Verschlechterungspfad, so auch $\gamma$. Ist $\gamma$ sogar ein Verbesserungspfad, so auch $\phi(\gamma)$.
\end{prop}

\begin{proof}
	Sei zunächst $\phi(\gamma)$ ein Nicht-Verschlechterungspfad. Angenommen $\gamma$ wäre kein Nicht-Verschlechterungspfad, dann gäbe es also ein $k$, sodass sich der im $k$-ten Schritt abweichende Spieler echt verschlechtert, d.h. $c_{i(k)}(x^k) > c_{i(k)}(x^{k-1})$. Da $\sigma: J \to I$ surjektiv ist, gibt es nun einen Spieler $j \in J$ mit $\sigma(j) = i(k)$ und wegen der Ordinalität von $\phi$ gilt für diesen, dass er sich im $k$-ten Schritt von $\phi(\gamma)$ echt verschlechtert, d.h. $d_j(\phi(x^k)) > d_j(\phi(x^{k-1}))$. 
	
	Insbesondere wissen wir damit, dass es auch in $\phi(\gamma)$ einen Spieler $j(k)$ geben muss, der im $k$-ten Schritt seine Strategie echt ändert. Damit gilt
		\[\phi_{\sigma(j(k))}\left(x_{\sigma(j(k))}^k\right) = \left(\phi(x^k)\right)_{\sigma(j(k))} \neq \left(\phi(x^{k-1})\right)_{\sigma(j(k))} = \phi_{\sigma(j(k))}\left(x_{\sigma(j(k))}^{k-1}\right)\]
	Daraus folgt $x_{\sigma(j(k))}^k \neq x_{\sigma(j(k))}^{k-1}$ und, da es in einem Pfad pro Schritt höchstens einen abweichenden Spieler geben kann, auch $i(k) = \sigma(j(k))$. Aus der Ordinalität von $\phi$ folgt dann aber aus $c_{i(k)}(x^k) > c_{i(k)}(x^{k-1})$, dass sich aus $j(k)$ echt verschlechtert, d.h. $d_{j(k)}(\phi(x^k)) > d_{j(k)}(\phi(x^{k-1}))$ - im Widerspruch dazu, dass $\phi(\gamma)$ ein Nicht-Verschlechterungspfad ist.

	Ist nun $\gamma$ ein Verbesserungspfad, gilt in jedem Schritt $c_{i(k)}(x^{k}) < c_{i(k)}(x^{k-1})$ und daher - mit dem gleichen Argument wie eben - auch $d_{j(k)}(\phi(x^{k+1})) < d_{j(k)}(\phi(x^k))$. Also ist auch $\phi(\gamma)$ ein Verbesserungspfad.
\end{proof}

Im Allgemeinen bilden (ordinale) Morphismen Pfade nicht wieder auf Pfade ab. Ist nämlich $(\sigma, \phi): \Gamma \to \Delta$ ein solcher Morphismus und gibt es zwei Spieler $j$ und $j'$ mit $\sigma(j) = \sigma(j') \eqqcolon i$, dann kann eine einseitige Abweichung von Spieler $i$ in $\Gamma$ zu einer gleichzeitigen Abweichung der \emph{beiden} Spieler $j$ und $j'$ in $\Delta$ führen. Dies kann allerdings nicht passieren, wenn die Spielerabbildung $\sigma$ injektiv ist. 

\begin{beob}\label{beob:PfadeAufPfade}
	Morphismen $(\sigma, \phi)$ mit injektivem $\sigma$ bilden Pfade auf Pfade ab.
\end{beob}

Dies erlaubt uns folgendes Korollar:

\begin{kor}\label{kor:ReflAbg}
	Sei $(\sigma, \phi): \Gamma \to \Delta$ ein ordinaler Morphismus mit bijektiver Spielerabbildung $\sigma$ und $\gamma = (x^0, \dots, x^n)$ ein Verbesserungspfad in $X$. Ist dann $\phi(\gamma)$ abgeschlossen in $\Delta$, so ist auch $\gamma$ in $\Gamma$ abgeschlossen.
\end{kor}
	
\begin{proof}	
	Nehmen wir im Widerspruch zur Behauptung an, $\gamma$ wäre nicht abgeschlossen. Dann gäbe es folglich ein Strategieprofil $x^{n+1} \in X$, welches $\gamma$ zu einem Verbesserungspfad $\gamma' \coloneqq (x^0, \dots, x^n, x^{n+1})$ in $\Gamma$ verlängert. Nach \Cref{beob:PfadeAufPfade} ist dann auch $\phi(\gamma')$ ein Pfad in $\Delta$ und mit \Cref{prop:NVReflVerbErh} sogar ein Verbesserungspfad, also insbesondere eine Verlängerung des Pfades $\phi(\gamma)$ - ein Widerspruch zu dessen Abgeschlossenheit.
\end{proof}

\begin{kor}\label{kor:ReflektNG}
	Ordinale Morphismen mit bijektiver Spielerabbildung reflektieren Nash-Gleichgewichte. Das heißt, ist $x \in X$ ein Strategieprofil in $\Gamma$ und $(\sigma, \phi)$ ein ordinaler Morphismus in ein Spiel $\Delta$, sodass $\phi(x)$ ein Nash-Gleichgewicht in diesem ist, dann war bereits $x$ ein Nash-Gleichgewicht von $\Gamma$.
\end{kor}

\begin{proof}
	Es ist $x$ ein trivialer Verbesserungspfad in $\Gamma$, dessen Bild $\phi(x)$ in $\Delta$ abgeschlossen ist. Daher ist mit \Cref{kor:ReflAbg} $x$ in $\Gamma$ abgeschlossen und folglich (vgl. \Cref{beob:VerbPfadeundNGe}) $x$ ein Nash-Gleichgewicht in $\Gamma$.
\end{proof}

\todo[inline]{Zu obigen Sätzen jeweils Beispiele für Notwendigkeit der Voraussetzungen finden}

Aus diesem Korollar folgt insbesondere, dass Nash-Gleichgewichte in von einem verallgemeinerten ordinalen Potential induzierten Koordinationsspiel auch Nash-Gleichgewichte im Ausgangsspiel sind.\todo{Näher ausführen - auf entsprechende Sätze im Potential-Kapitel verweisen}

\begin{bsp}
	Ordinale Isomorphismen haben insbesondere injektive Spieler- und Strategieabbildungen. Damit erfüllen sie die Voraussetzungen von \Cref{kor:ReflektNG}, d.h. ordinal isomorphe Spiele haben die gleichen Nash-Gleichgewichte.
\end{bsp}

Dies wiederum zeigt, dass ordinale Potentiale lokale Nash-Potentiale sind.


\subsection{Weitere Sätze mit Morphismen}

\todo[inline]{Weitere unsortierte Sätze...:}


\begin{prop}\label{prop:DummySpielerWeglassen}
	Sei $(\sigma, \phi): \Gamma \to \Delta$ ein biordinaler\todo{Begriff definieren!}{} Morphismus mit injektiver Spielerabbildung und surjektiven Strateegieabbildungen. Seien ferner alle Spieler in $I\setminus \sigma(J)$ Dummy-Spieler\todo{Begriff definieren! Evtl. zusammen mit Dummy-Spielen}. Dann ist ein Strategieprofil $x \in X$ genau dann ein Nash-Gleichgewicht in $\Gamma$, wenn $\phi(x)$ ein Nash-Gleichgewicht in $\Delta$ ist.
\end{prop}

\todo[inline]{Kann man das noch schöner ausdrücken (mit Retrakten o.ä.)?}

\begin{proof}
	Sei zunächst $x \in X$ ein Nash-Gleichgewicht in $\Gamma$ und $y_j \in Y_j$. Da $\phi_j$ surjektiv ist, gibt es dann ein $\hat{x}_{\sigma(j)}$ mit $\phi_j(\hat{x}_{\sigma(j)}) = y_j$. Wegen der Injektivität von $\sigma$ gilt ferner $\phi(x \mid \hat{x}_{\sigma(j)}) = \left(\phi(x) \mid y_j\right)$.
	
	Da $x$ ein Nash-Gleichgewicht ist, gilt nun $c_{\sigma(j)}(x) \leq c_{\sigma(j)}(x \mid \hat{x}_{\sigma(j)})$ und folglich wegen der Biordinalität von $\phi$ auch $d_j(\phi(x)) \leq d_j(\phi(x \mid \hat{x}_{\sigma(j)})) = d_j(\phi(x) \mid y_j)$. Also ist auch $\phi(x)$ ein Nash-Gleichgewicht in $\Delta$.
	
	Ist umgekehrt $\phi(x)$ ein Nash-Gleichgewicht in $\Delta$ und $\hat{x}_i$ eine alternative Strategie von Spieler $i \in I$. 
	\begin{description}
		\item[1. Fall:] Liegt $i$ in $I\setminus \sigma(J)$, so ist $i$ nach Voraussetzung ein Dummy-Spieler und folglich gilt: $c_i(x) = c_i(x \mid \hat{x}_i)$.
		\item[2. Fall:] Liegt $i$ hingegen in $\sigma(J)$, so gibt es also einen Spieler $j \in J$ mit $\sigma(j) = i$. Aus der Injektivität von $\sigma$ und dem Wissen, dass $\phi(x)$ ein Nash-Gleichgewicht ist, folgt nun $d_j(\phi(x \mid \hat{x}_i)) = d_j(\phi(x) \mid \phi_j(\hat{x}_{\sigma(j)})) \geq d_j(\phi(x))$. Über die Biordinalität von $\phi$ erhalten wir schließlich $c_i(x \mid \hat{x}_i) = c_{\sigma(j)}(x \mid \hat{x}_i) \geq c_{\sigma(j)}(x) = c_i(x)$. 
	\end{description}
	Insgesamt ist also auch $x$ ein Nash-Gleichgewicht.
\end{proof}

Mit dieser Proposition können wir nun die in \Cref{beob:endlicheSpiele} gemachte Beobachtung formalisieren (und beweisen), dass Spieler mit nur einer einzigen Strategie beim Untersuchen von Nash-Gleichgewichten immer ignoriert werden können:

\begin{bsp}\label{bsp:EinStratSpielerWeglassen}
	Spieler mit nur einer einzigen Strategie können immer entfernt werden ohne die Nash-Gleichgewichte zu verändern.
	
	Sei dazu $\Gamma$ ein beliebiges Spiel und $I' \subseteq I$ eine solche Teilmenge von Spielern, dass alle Spieler $i$ in $I \setminus I'$ genau eine Strategie $\ast_i$ besitzen. Dann betrachten wir die Einschränkung von $\Gamma$ auf die Spieler in $I'$, nämlich $\Gamma' \coloneqq (I', \prod_{i \in I'}X_i, (c'_i)_{i \in I'})$, wobei die Kostenfunktionen wie folgt definiert sind:
		\[c'_{\hat{\imath}}(x') \coloneqq c_{\hat{\imath}}(x), \text{ wobei $x \in X$ definiert ist durch } x_i = \begin{cases}x'_i, &i \in I'\\ \ast_i, &\text{sonst}\end{cases}\]
	und den folgenden Morphismus $(\sigma, \phi): \Gamma \to \Gamma'$ zwischen den beiden Spielen:
	\begin{align*}
		\sigma:	&&I\setminus I'	\to	 I:	&&i	\mapsto	i  \\
		\phi_i:	&&X_i	\to	 X_i:		&&x_i	\mapsto	x_i
	\end{align*}	
	Dieser ist offenbar kostenerhaltend, also insbesondere biordinal, und $I\setminus \sigma(I') = I \setminus I'$ enthält nur Dummy-Spieler. Damit folgt die Behauptung aus \Cref{prop:DummySpielerWeglassen}.
\end{bsp}

\todo[inline]{Kein Zusammenhang zum Vorhergebenden:}

\begin{kor}
	Seien $\Gamma$ und $\Gamma'$ zwei ordinal-isomorphe Spiele. Dann hat $\Gamma$ genau dann die FIP, wenn $\Gamma'$ diese besitzt.
\end{kor}

Eine Verallgemeinerung von \Cref{beob:ZshExGewPot} lautet

\begin{prop}
	Seien $\Gamma$ und $\Delta$ zwei Spiele. Dann gibt es genau dann einen gewichteten Morphismus $\Gamma \to \Delta$, wenn es Gewichte $(w_i)_{i\in I}$ und einen exakten Morphismus $\Gamma \to w_i \cdot \Delta$ gibt.
\end{prop}

\begin{kor}
	Zwei Spiele $\Gamma$ und $\Delta$ sind genau dann gewichtet isomorph, wenn ... \todo{usw.}
\end{kor}

\todo[inline]{Ist der folgende Satz hilfreich? Wenn ja, wo unterbringen (eig. kein Potentialsatz!)}

\begin{prop}
	Sei $(\sigma, \phi): \Gamma \to \Delta$ ein monotoner Morphismus und $\gamma = (x^0, x^1, \dots)$ ein Pfad in $X$, sodass auch $\phi(\gamma) \coloneqq (\phi(x^0), \phi(x^1), \dots)$ ein Pfad in $Y$ ist. Ist dann $\gamma$ ein Nicht-Verschlechterungspfad, so auch $\phi(\gamma)$. Ist $\phi(\gamma)$ sogar ein Verbesserungspfad, so auch $\gamma$.
\end{prop}

\begin{proof}
	Zunächst beobachten wir, dass $\sigma$ immer den in $\phi(\gamma)$ im $k$-ten Schritt abweichenden Spieler (sofern es einen solchen gibt) auf den im gleichen Schritt abweichenden Spieler in $\gamma$ abbildet. Ist nämlich $j(k)$ der entsprechende Spieler aus $\Delta$, so gilt:
	\[\phi_{\sigma(j(k))}\left(x_{\sigma(j(k))}^k\right) = \left(\phi(x^k)\right)_{\sigma(j(k))} \neq \left(\phi(x^{k-1})\right)_{\sigma(j(k))} = \phi_{\sigma(j(k))}\left(x_{\sigma(j(k))}^{k-1}\right)\]
	Daraus folgt $x_{\sigma(j(k))}^k \neq x_{\sigma(j(k))}^{k-1}$ und, da es in einem Pfad pro Schritt höchstens einen abweichenden Spieler geben kann, auch $i(k) = \sigma(j(k))$.
	
	Ist nun $\gamma$ ein Nicht-Verschlechterungspfad und es gibt im $k$-ten Schritt einen (echt) abweichenden Spieler $j(k)$, so ist also $\sigma(j(k)) = i(k)$ und daher $c_{\sigma(j(k))}(x^k) \leq c_{\sigma(j(k))}(x^{k-1})$. Mit der Monotonie von $\phi$ folgt daraus direkt auch $d_{j(k)}(\phi(x^k)) \leq d_{j(k)}(\phi(x^{k-1}))$.
	
	Ist $\phi(\gamma)$ ein Verbesserungspfad, gilt in jedem Schritt $d_{j(k)}(\phi(x^{k+1})) < d_{j(k)}(\phi(x^k))$ und daher - erneut wegen der Monotonie von $\phi$ - auch $c_{\sigma(j(k))}(x^{k+1}) < c_{\sigma(j(k))}(x^{k})$.
\end{proof}
