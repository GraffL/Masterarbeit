\section{Morphismen von Spielen}

\subsection{Definitionen}

\begin{defn}
	Ein \emph{Spiel in strategischer Form $\Gamma$} ist gegeben durch ein Tupel $(I, X = \prod_{i \in I}X_i, (K_i)_{i\in I}, (u_i: X \to K_i)_{i\in I})$. Dabei ist:
	\begin{itemize}
		\item $I$ die Menge der Spieler,
		\item $X_i$ die Menge der (reinen) Strategien von Spieler $i$,
		\item $K_i$ die \todo[inline]{Übersetzung von Payoff-Raum evtl. angepasst für Kosten?} von Spieler $i$ und
		\item $u_i$ die Kostenfunktion von Spieler $i$
	\end{itemize}
\end{defn}

\begin{defn}
	Zwei Spiele $\Gamma = (I, X, (K_i)_{i\in I}, (u_i)_{i\in I})$ und $\Gamma' = (I, X', (K_i)_{i\in I}, (u'_i)_{i\in I})$ heißen \emph{äquivalent}, wenn es für jeden Spieler $i$ eine bijektive Abbildung $\phi_i: X_i \to X'_i$ gibt, sodass gilt:
		\[\forall x \in X: u_i(x) = u'_i(\phi(x)) \]
	\todo{Permutation von Spielern erlauben?}
\end{defn}

\begin{bem}
	Zwei Spiele sind also genau dann äquivalent, wenn sie sich ausschließlich durch Umbenennung der Strategien ineinander überführen lassen. In \cite{MonShap} (S. 133) wird dies als Isomorphie von Spielen bezeichnet.
\end{bem}

\begin{defn}\label{def:SpielIsomLap}
	Zwei Spiele $\Gamma = (I, X, (K_i)_{i\in I}, (u_i)_{i\in I})$ und $\Gamma' = (I, X', (K'_i)_{i\in I}, (u'_i)_{i\in I})$ heißen \emph{isomorph}, falls es bijektive Abbildungen $\phi_i: X_i \to X'_i$ sowie bijektive und monotone Abbildungen $\psi_i: K_i \to K'_i$ gibt, sodass alle Diagramme der folgenden Form kommutieren:
	
	\begin{center}
			\begin{tikzcd}
			X \arrow{r}{\phi}\arrow{d}{u_i} & X' \arrow{d}{u'_i} \\
			K_i \arrow{r}{\psi_i}			& K'_i
		\end{tikzcd}
	\end{center}
\end{defn}

\begin{bem}
	Diese Definition ergibt sich aus der abstrakteren Definition für \todo{...} in \cite{LapGameCat}.
\end{bem}

\begin{defn}
	Zwei im Sinne von \cref{def:SpielIsomLap} isomorphe Spiele heißen \emph{sozial isomorph}, wenn zusätzlich die Funktion
		\[\sum \psi_i: \prod_{i \in I}K_i \to \prod_{i \in I} K'_i \]
	monoton ist. \todo{Das macht natürlich nur Sinn, wenn auf den beiden Produkträumen auch totale (?) Ordnungen existieren}
\end{defn}

\begin{bsp}
	lineare Funktionen
\end{bsp}

\begin{defn}\label{def:NashMorphismus}
	Ein \emph{Nash-Morphismus} $\gamma: \Gamma \to \Gamma'$ zwischen zwei Spielen $\Gamma = (I, X, (K_i)_{i\in I}, (u_i)_{i\in I})$ und $\Gamma' = (I, X', (K'_i)_{i\in I}, (u'_i)_{i\in I})$ ist gegeben durch Abbildungen $\phi_i: X_i \to X'_i$ sodass gilt:
		\[\forall x\in X, i \in I, \hat{x}_i \in X_i: u_i(\hat{x}_i,x_{-i}) > u_i(x) \Rightarrow u'_i(\phi(\hat{x}_i,x_{-i})) > u'_i(\phi(x))\]
	Der Morphismus $\gamma$ heißt \emph{Nash-Isomorphismus} (und die beiden Spiele dann \emph{Nash-isomorph}), wenn die $\phi_i$ bijektiv sind und gilt:
		\[\forall x\in X, i \in I, \hat{x}_i \in X_i: u_i(\hat{x}_i,x_{-i}) > u_i(x) \iff u'_i(\phi(\hat{x}_i,x_{-i})) > u'_i(\phi(x))\]
\end{defn}

\begin{beob}
	Es gilt:
	\begin{center}
		äquivalent $\Rightarrow$ sozial isomorph $\Rightarrow$ isomorph $\Rightarrow$ Nash-isomorph
	\end{center}
\end{beob}


\subsection{Erste Sätze}

\begin{lemma}
	Seien $\Gamma$ und $\Gamma'$ zwei Nash-isomorphe Spiele. Dann ist $x \in X$ genau dann ein Nashgleichgewicht von $\Gamma$, wenn $\phi(x) \in X'$ ein Nashgleichgewicht von $\Gamma'$ ist.
\end{lemma}

\begin{proof}.
	
	\todo[inline]{folgt direkt mit Definitionen}
\end{proof}

\begin{lemma}
	Seien $\Gamma$ und $\Gamma'$ zwei Nash-isomorphe Spiele. Dann hat $\Gamma$ genau dann die FIP, wenn $\Gamma'$ diese besitzt.
\end{lemma}

\begin{proof}.
	
	\todo[inline]{Beweis über Verbesserungspfad im einen entspricht Verbesserungspfad im anderen. Evtl. direkt das als Lemma formulieren und dann die beiden vorherigen Lemmas als Korollare daraus?}
\end{proof}

\begin{lemma}
	Sei $\gamma: \Gamma \to \Gamma'$ ein Nash-Morphismus und $x \in X$. Ist dann $\phi(x) \in X'$ ein Nashgleichgewicht von $\Gamma'$, so ist auch $x$ selbst schon ein Nashgleichgewicht (von $\Gamma$).
\end{lemma}

\begin{proof}
	.
	
	\todo[inline]{Nachrechnen - evtl. mit vorherigen Sätzen verbinden bzw. schon davor zeigen, damit diese ein Korollar werden?}
\end{proof}

\begin{lemma}
	Seien $\Gamma$ und $\Gamma'$ zwei sozial isomorphe Spiele. Dann ist $x \in X$ genau dann ein soziales Optimum von $\Gamma$, wenn $\phi(x) \in X'$ ein soziales Optimum von $\Gamma'$ ist.
\end{lemma}

\begin{proof}
	.
	
	\todo[inline]{Folgt direkt mit Definitionen}
\end{proof}

\begin{satz}
	Besitzt ein Spiel $\Gamma$ ein ordinales Potential, so ist es isomorph zu einem Auslastungsspiel.
\end{satz}

\begin{proof}
	Analog zum Beweis der Äquivalenz von Spielen mit exaktem Potential und Auslastungsspielen in \cite{MonShap}, Beweis orientiert sich an \cite{MultiPotGames}.
\end{proof}

\begin{beob}
	Besitzt ein Spiel ein verallgemeinertes ordinales Potential, so gibt es einen Nash-Morphismus in/von \todo{Was von beidem?} ein Auslastungsspiel.
\end{beob}

\begin{proof}
	.
	\todo[inline]{Proofmining in oberem Beweis}
\end{proof}

\begin{beob}
	Nach \cite{MonShap} Lemma 2.5 hat jedes Spiel mit FIP ein verallgemeinertes Potential, also \todo[inline]{in/von...}
\end{beob}