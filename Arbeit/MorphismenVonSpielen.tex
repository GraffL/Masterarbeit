\section[Morphismen]{Morphismen und Isomorphismen von Spielen}\label{sec:Morphismen}

\todo[inline]{Motivation: Warum betrachtet man überhaupt Morphismen von Spielen. Wie induzieren diese Isomorphismen?}

Mögliche Motivationen:
\begin{itemize}
	\item Jedes Potential definiert selbst wieder ein Spiel mit einer gemeinsamen Kostenfunktion für alle Spieler. Und in einem gewissen Sinne ist dieses Spiel \glqq äquivalent\grqq{} zum ursprünglichen Spiel (z.B. gleiche Gleichgewichtspunkte). Das Hin- und Herwechseln zwischen diesen beiden Versionen eines Spiels kann man durch Morphismen beschreiben und die Äquivalenz der beiden wird dann dadurch sichtbar, dass diese Morphismen \emph{Isomorphismen} sind.
	\item Die Äquivalenz zwischen exakten Potentialspielen und Auslastungsspielen wird durch Morphismen beschrieben.
	\item Kategorientheoretische Sicht: Um eine Kategorie (hier: die der Spiele) zu verstehen, muss man ihre Morphismen kennen. Kennt man diese, so ergeben sich aus diesen auf natürliche Weise weitere Begriffe wie Isomorphismen von Spielen, Summen oder Produkte von Spielen.
\end{itemize}

\todo[inline]{Diese Punkte (insbesondere den letzten) näher ausführen?}

\subsection{Definitionen}

Zu zwei gegebenen Spielen $\Gamma = (I, X, (c_i)_{i\in I})$ und $\Delta = (J, Y, (d_j)_{j\in J})$ kann man wie folgt eine Abbildung $(\sigma, \phi): \Gamma \to \Delta$ zwischen diesen beiden definieren:
\begin{itemize}
	\item Eine Abbildung $\sigma: J \to I$ zwischen den Spielermengen und
	\item für jeden Spieler $j \in J$ eine Abbildung $\phi_j: X_{\sigma(j)} \to Y_j$ seiner Strategien.
\end{itemize}

Diese Art und Weise Abbildungen zwischen zwei Spielen zu definieren ergibt sich aus dem - noch deutlich allgemeineren - Ansatz hierzu von \citeauthor{LapGameCat} in \cite{LapGameCat}\footnote{\citeauthor{LapGameCat} lässt in seiner Definition allerdings alle beteiligten Abbildungen in die jeweils umgekehrte Richtung gehen. Wir betrachten hier also gewissermaßen die zur dort definierten duale Kategorie}. Hierbei orientieren wir uns vor allem an dem dort in Kapitel 4 vorgestellten Morphismusbegriff für topologische Spiele (in denen die Spielermenge ein topologischer Raum und der Strategieraum eine Garbe über diesem ist) bzw. dessen Spezialisierung für Spiele mit diskreter Spielermenge aus Kapitel 5. 

In diesem Kontext ergibt es sich etwas natürlicher, dass die Abbildung zwischen den Spielermengen in die entgegengesetzte Richtung zu der zwischen den Strategieräumen verläuft. Eine andere, auch im hier vorliegenden Kontext sinnvolle Begründung hierfür liefert aber die folgende Bemerkung:

\begin{bem}
	Alle Strategieabbildungen $\phi_j$ zusammen induzieren eine \emph{Strategieprofilabbildung}
	\[\phi: X \to Y: x=(x_i)_{i\in I} \mapsto \phi(x) := \left(\phi_j(x_{\sigma(j)})\right)_{j \in J} \]
\end{bem}

Würde die Abbildung $\sigma$ zwischen den Spielermengen in die andere, \glqq natürlichere\grqq{} Richtung verlaufen (also von $I$ nach $J$), so müssten wir zusätzlich fordern, dass diese bijektiv ist, damit in der obigen Form Strategieprofile auf Strategieprofile abgebildet werden (vgl. etwa \cite{CatGameTheory}). Denn nur dann wäre $\phi(x) \coloneqq \left(\phi_i(x_i)\right)$ wieder ein vollständiges Strategieprofil in $Y$.

\begin{bem}\todo{Genauer beschreiben...}
	Möchte man trotzdem Spieler- und Strategieabbildungen in die gleiche Richtung haben, so kann man einem alternativen Ansatz zur Definition von Morphismen von Spielen folgen, welcher von \citeauthor{Foundations} in \cite{Foundations} verwendet wird. Darin werden einzelne Strategien nicht zwangsläufig wieder auf einzelne Strategien abgebildet, sondern können gleich auf ganze Teilmengen des Bildstrategieraums abgebildet werden. Von diesem Morphismentyp werden dann verschiedene \glqq approximativ kostenerhaltende\grqq{} Varianten betrachtet und die sich dadurch ergebende Kategorie studiert.
\end{bem}

\begin{notation}
	In den meisten Fällen wird die Abbildung zwischen den Spielermengen bijektiv sein. In diesen Fällen werden wir zur Vereinfachung der Notation ohne Einschränkung davon ausgehen, dass die Spielermengen beider an der Abbildung beteiligten Spiele bereits gleich und geeignet permutiert sind, sodass $\sigma$ die Identitätsabbildung ist. Damit kann diese in der Notation weggelassen werden und die Abbildung zwischen den Spielen besteht nur noch aus den Abbildungen $\phi_i: X_i \to Y_i$ zwischen den Strategieräumen.
\end{notation}

Abbildungen der obigen Form nehmen noch keinerlei Rücksicht auf die Kostenfunktionen der jeweiligen Spiele. Da diese aber in der Regel die interessierenden Eigenschaften eines Spiels (wie beispielsweise Gleichgewichte) festlegen, werden derartige Abbildungen im Allgemeinen noch wenig Aussagen über die beteiligten Spiele ermöglichen. So besagt der durch diese Art Abbildungen induzierte Isomorphiebegriff bspw. nur, dass zwei Spiele Spieler- und Strategiemengen gleicher Kardinalität besitzen.

Echte Morphismen zwischen Spielen sollten folglich noch mehr der Struktur eines Spiels erhalten, insbesondere in irgendeiner Form \glqq verträglich\grqq{} mit den Kostenfunktionen sein. Je nach dem, welche Eigenschaften die Morphismen (und insbesondere die dadurch induzierten Isomorphismen) erhalten sollen, erhält man so unterschiedlich starke Verträglichkeitsbedingungen. Einige Möglichkeiten dafür werden wir nun kennenlernen.

Eine relative starke Forderung ist die, dass Morphismen \emph{kostenerhaltend} sein müssen, wie sie in \cite{ReprOfFiniteGamesAsNCG} gestellt wird:

\begin{defn}
	Ein Morphismus $(\sigma, \phi_j)$ von $\Gamma = (I, X, (c_i)_{i\in I})$ nach $\Delta = (J, Y, (d_j)_{j\in J})$ heißt \emph{kostenerhaltend}, wenn für alle Strategieprofile $x \in X$ und jeden Spieler $j \in J$ gilt:
		\[c_{\sigma(j)}(x) = d_j(\phi(x)) \]
	Ist ein solcher Morphismus gleichzeitig ein Isomorphismus, so nennen wir $\Gamma$ und $\Delta$ \emph{äquivalent}.
\end{defn}

\begin{bem}
	Zwei Spiele sind also genau dann äquivalent, wenn sie sich ausschließlich durch Umbenennung der Strategien ineinander überführen lassen. In \cite{MonShap} (S. 133) wird dies als Isomorphie von Spielen bezeichnet.
\end{bem}

\begin{bsp}\label{bsp:Koordinationsspiel}
	Ein Spiel $\Delta = (J, Y, (d_j)_{j \in J})$ ist genau dann ein Koordinationsspiel, wenn es einen kostenerhaltenden Morphismus mit surjektiver Strategieprofilabbildung von einem 1-Personenspiel nach $\Delta$ gibt.
	
	Ist nämlich $\Delta$ ein Koordinationsspiel, so definiert man ein 1-Personenspiel $K \coloneqq (\{\ast\}, X, (c_\ast))$ mit $X = X_\ast \coloneqq Y$ und $c_\ast(y) \coloneqq d_{\hat{j}}(y)$ für einen beliebigen, aber festen Spieler $\hat{j} \in J$. Ferner definiert man den folgenden Morphismus $(\sigma, \phi): K \to \Delta$:
	\begin{align*}
		\sigma:	&&J		\to	 \{\ast\}:	&&j	\mapsto	\ast  \\
		\phi_i:	&&X_\ast	\to	 Y_j:	&&y	\mapsto	y_j
	\end{align*}	
	Dieser ist kostenerhaltend, denn für jedes Strategieprofil $y \in X = Y$ und jeden Spieler $j \in J$ gilt:
	\[c_{\sigma(j)}(y) = c_\ast(y) = d_{\hat{j}}(y) \overset{\Gamma \text{ Koord.spiel}}{=} d_j(y) = d_j(\phi(y))\]
	Zudem ist offenbar die Strategieprofilabbildung $\phi$ surjektiv.
	
	Sind umgekehrt ein 1-Personenspiel $K = (\{\ast\}, X, (c_\ast))$ sowie ein kostenerhaltender Morphismus  $(\sigma, \phi): K \to \Delta$ mit surjektiven Strategieabbildungen gegeben, dann ist $\Delta$ bereits ein Koordinationsspiel. Denn aufgrund der Surjektivität von $\phi$ gibt es zu jedem Strategieprofil $y \in Y$ ein Strategieprofil $x \in X$ mit $\phi(x) = y$. Da $(\sigma, \phi)$ außerdem kostenerhaltend ist, folgt dann für je zwei Spieler $j, \hat{j} \in J$: 
		\[d_{\hat{j}}(y) = d_{\hat{j}}(\phi(x)) = c_{\sigma(\hat{j})}(x) = c_\ast(x) = c_{\sigma(j)}(x) = d_j(\phi(x)) = d_j(y)\]
	Also ist $\Gamma$ tatsächlich ein Koordinationsspiel.
\end{bsp}

Dieses Beispiel formalisiert die Intuition, dass in einem Koordinationsspiel alle Spieler ein gemeinsames Ziel haben und daher zusammen \glqq wie ein Spieler\grqq{} spielen (d.h. das Koordinationsspiel kann durch ein 1-Personenspiel simuliert werden).\todo{Stimmt das so?}

\begin{bsp}.
	
	\todo[inline]{Kann man auch Dummy-Spiele auf derartige Weise beschreiben?}
\end{bsp}

\begin{bsp}
	Gegeben zwei Spiele $\Gamma = (I, X, (c_i)_{i\in I})$ und $\Delta = (J, Y, (d_j)_{j\in J})$, so können wir daraus neue Spiele konstruieren: Das Produkt bezüglich kostenerhaltenden Morphismen: $\Gamma \times \Delta \coloneqq (K \coloneqq I \dot{\cup} J, X\times Y, (e_k)_{k \in K})$. Dabei ist $e_k(x,y) \coloneqq c_k(x)$ für $k \in K$ und analog für $k \in J$. \todo{Gibt es irgendeine anschauliche Konstruktion von Spielen, die sich als Produkt beschreiben lässt?}
\end{bsp}


In \cite{LapGameCat} stellt \citeauthor{LapGameCat} folgende schwächere Verträglichkeitsbedingung:

\begin{defn}
	Ein Morphismus $(\sigma, \phi_j)$ von $\Gamma = (I, X, (c_i)_{i\in I})$ nach $\Delta = (J, Y, (d_j)_{j\in J})$ heißt \emph{monoton}, wenn es für jeden Spieler $j \in J$ eine montone Abbildung $f_j: \IR \to \IR$ gibt, sodass das folgende Diagramm kommutiert:	
	\begin{center}
		\begin{tikzcd}
			X \rar{\phi} \dar{c_{\sigma(j)}} & Y \dar{d_j} \\
			\IR \rar{f_j}		& \IR
		\end{tikzcd}
	\end{center}
\end{defn}

\begin{bem}
	Eine derartige Abbildungen $f_j$ für einen Spieler $j \in J$ gibt es genau dann, wenn für je zwei Strategieprofile $x, \hat{x} \in X$ gilt:
		\begin{align}\label{eq:CharExMonMor}
			c_{\sigma(j)}(x) \leq c_{\sigma(j)}(\hat{x}) \implies d_j(\phi(x)) \leq d_j(\phi(\hat{x}))
		\end{align}
\end{bem}

\begin{proof}
	Existieren derartige Abbildungen $f_i$, so gilt für zwei Strategieprofile $x, \hat{x} \in X$ mit $c_{\sigma(j)}(x) \leq c_{\sigma(j)}(\hat{x})$:
		\[d_j(\phi(x)) = f_j\circ c_{\sigma(j)}(x) \leq f_j\circ c_{\sigma(j)}(\hat{x}) = d_j(\phi(\hat{x}))\]
		
	Gilt umgekehrt die Eigenschaft \ref{eq:CharExMonMor}, so können wir wie folgt monotone Abbildungen $f_j: \IR \to \IR$ definieren:
		\[f_j: \IR \to \IR: t \mapsto \begin{cases}
			\sup \Set{d_j(\phi(x)) | x \in X},								&\not\exists x \in X: c_{\sigma(j)}(x) \geq t \\
			\inf\, \Set{d_j(\phi(x)) | x \in X, c_{\sigma(j)}(x) \geq t}, 	&t \geq 0 \text{ und } \exists x \in X: c_{\sigma(j)}(x) \geq t\\
			\sup \Set{d_j(\phi(x)) | x \in X, c_{\sigma(j)}(x) \leq t}, 	&t <    0 \text{ und } \exists x \in X: c_{\sigma(j)}(x) \leq t\\
			\inf\,\, \Set{d_j(\phi(x)) | x \in X},							&\not\exists x \in X: c_{\sigma(j)}(x) \leq t 
		\end{cases} \]
	\todo[inline]{Eigenschaften nachweisen}
\end{proof}

\begin{prop}
	Ist ein Spiel monoton isomorph zu einem Koordinationsspiel, so besitzt es ein skaliertes Potential.
\end{prop}

\begin{proof}.
	
	\todo[inline]{Beweis... (stimmt die Aussage überhaupt? Evtl. nur ordinales Potential?)}
\end{proof}

\begin{kor}
	Sei $(\sigma, \phi): K \to \Gamma$ ein monotoner Morphismus mit surjektiver Strategieprofilabbildung von einem $1$-Personenspiel nach $\Gamma$. Dann besitzt $\Gamma$ ein skaliertes Potential.
\end{kor}

\begin{proof}.
	
	\todo[inline]{Mit \Cref{bsp:Koordinationsspiel}?}
\end{proof}

Dies ist also bereits ein Beispiel dafür, wie man die Existenz eines Potentials durch die Existenz eines passenden Isomorphismus zu einem Koordinationsspiel ausdrücken kann. Allerdings ist die Verträglichkeitsbedinung an monotone Morphismen noch zu stark um auf diesem Weg auch eine notwendige Bedingung für die Existenz eines skalierten\todo{ordinalen?}{} zu erhalten. Dies liegt daran, dass hier eine globale Montonie gefordert wird, während die meisten Potentiale lediglich lokale Monotonie benötigen.

Wir schränken nun also die Verträglichkeitsbedingung auf Nachbarschaften ein und kommen so zum Begriff eines ordinalen Morphismus. Einen entsprechenden (Iso-)Morphismusbegriff für exakte Potentiale definiert \citeauthor{ReprOfFiniteGamesAsNCG} in \cite{ReprOfFiniteGamesAsNCG}. Und analog hierzu lassen sich auch für die anderen Potentialbegriffe passende Begriffe eines Morphismus (und damit eines Isomorphismus) definieren:

\begin{defn}\label{def:PotentialMorphismen}
	Ein Morphismus $(\sigma, \phi)$ von $\Gamma = (I, X, (c_i)_{i\in I})$ nach $\Delta = (J, Y, (d_j)_{j\in J})$ heißt
	\begin{itemize}
		\item \emph{exakt}, wenn für alle Strategieprofile $x \in X$, Spieler $j \in J$ und Strategien $\hat{x}_{\sigma(j)}$ gilt:
		\[c_{\sigma(j)}(x) - c_{\sigma(j)}(x \mid \hat{x}_{\sigma(j)}) = d_j(\phi(x)) - d_j(\phi(x \mid \hat{x}_{\sigma(j)}))\]
		\item \emph{gewichtend}, wenn es einen Vektor $(w_j)_{j \in J} \in \IR_{\geq 0}^J$ gibt, sodass für alle Strategieprofile $x \in X$, Spieler $j \in J$ und Strategien $\hat{x}_{\sigma(j)}$ gilt:
		\[c_{\sigma(j)}(x) - c_{\sigma(j)}(x \mid \hat{x}_{\sigma(j)}) = w_j\cdot\left(d_j(\phi(x)) - d_j(\phi(x \mid \hat{x}_{\sigma(j)}))\right)\]
		\item \emph{skalierend}, wenn es streng monotone Funktionen $f_j$ gibt, sodass für alle Strategieprofile $x \in X$, Spieler $j \in J$ und Strategien $\hat{x}_{\sigma(j)}$ gilt:
		\[c_{\sigma(j)}(x) - c_{\sigma(j)}(x \mid \hat{x}_{\sigma(j)}) = f_j(d_j(\phi(x))) - f_j(d_j(\phi(x \mid \hat{x}_{\sigma(j)})))\]
		\item \emph{ordinal}, wenn für alle Strategieprofile $x \in X$, Spieler $j \in J$ und Strategien $\hat{x}_{\sigma(j)}$ gilt:
		\[c_{\sigma(j)}(x) \leq c_{\sigma(j)}(x \mid \hat{x}_{\sigma(j)}) \implies d_j(\phi(x)) \leq d_j(\phi(x \mid \hat{x}_{\sigma(j)}))\]
		\item \emph{beste Antwort-erhaltend}, wenn für alle Spieler $j \in J$ und Strategieprofile $x_{-\sigma(j)} \in X_{-\sigma(j)}$ gilt:
		\[\phi(\arg \min_{x_{\sigma(j)} \in X_{\sigma(j)}}c_{\sigma(j)}(x)) \subseteq \arg \min_{y_j \in Y_j} d_j(\phi_{-\sigma(j)}(x \mid y_j)\]
	\end{itemize}	
\end{defn}

\begin{beob}
	Die Verknüpfung zweier \_-Morphismen ist nicht notwendigerweise wieder ein \_-Morphismus. 
	
	\todo[inline]{Beispiel!}
	
	Die Aussage gilt aber, wenn 
	
	\todo[inline]{einer der beiden/beide bijektiv auf ... sind?}
\end{beob}

Es ist nun leicht zu sehen, dass diese Isomorphismusbegriffe tatsächlich dazu definiert werden können, die Existenz der verschiedenen Potentiale zu definieren:

\begin{bem}
	Ein Spiel $\Delta$ besitzt genau dann ein exaktes/gewichtetes/ordinales/Beste-Antwort-Potential, wenn es exakt/gewichtet/ordinal/beste Antwort-isomorph zu einem Koordinationsspiel ist.
\end{bem}\todo{Beweis?}

\todo[inline]{Zu beste Antwort: vgl. \cite{FictPlayProp}, \cite{BestRespEq}}

\begin{beob}
	Es gelten folgende Beziehungen zwischen den verschiedenen Morphismenbegriffen:
	\begin{center}
		\begin{tikzpicture}
			\node[] (ke) {kostenerhaltend};
			\node[right of=ke, node distance=7em] (ex) {exakt};
			\node[right of=ex, node distance=7em] (gew) {gewichtet};
			\node[right of=gew, node distance=7em] (sk) {skaliert};
			\node[right of=sk, node distance=7em] (ord) {ordinal};
			
			\node[above of=gew, node distance=2em] (mon) {monoton};
	
			\node[below of=ord, node distance=2em] (BA) {Beste Antwort};
						
			\draw[-implies, double equal sign distance] (ke) -- (ex);
			\draw[-implies, double equal sign distance] (ex) -- (gew);
			\draw[-implies, double equal sign distance] (gew) -- (sk);
			\draw[-implies, double equal sign distance] (sk) -- (ord);
			\draw[-implies, double equal sign distance] (ke) -- (mon.west);
			\draw[-implies, double equal sign distance] (mon.east) -- (sk);
			\draw[-implies, double equal sign distance] (sk) -- (BA.west);
		\end{tikzpicture}
	\end{center}
\end{beob}

\begin{lemma}
	Ein Spiel $\Delta$ besitzt genau dann ein exaktes/gewichtetes/ordinales/verallgemeinertes ordinales Potential, wenn es einen exakten/gewichteten/biordinalen/ordinalen Morphismus mit surjektiver Strategieprofilabbildung von einem 1-Personenspiel nach $\Delta$ gibt.
\end{lemma}

\begin{proof}
	Sei zunächst $\Delta$ ein Spiel mit einem \sPot{} $P$. 
	\todo[inline]{Beweis einfügen}
\end{proof}




\subsection{Beweise der Potentialsätze}

\todo[inline]{Ordinale Morphismen reflektieren Verbesserungspfade:}

\begin{prop}
	Sei $(\sigma, \phi): \Gamma \to \Delta$ ein ordinaler Morphismus und $(x^0, x^1, \dots)$ eine Folge in $X$. Ist $\phi(x^0), \phi(x^1), \dots$ ein Verbesserungspfad in $\Delta$, so auch $(x^0, x^1, \dots)$ in $\Gamma$.\todo{Abgeschlossenheit in die andere Richtung?} 
\end{prop}

\begin{proof}
	Da $\sigma$ injektiv ist, bleibt die Eigenschaft unilateralen Abweichung erhalten. Die Ordinalität des Morphismus stellt ferner sicher, dass jeder Verbesserungsschritt ein Verbesserungsschritt bleibt.
	
	Nehmen wir nun an $\phi(x^0), \dots, \phi(x^n)$ wäre ein abgeschlossener Verbesserungspfad in $\Delta$, aber $x^0, \dots, x^n$ nicht abgeschlossen in $\Gamma$. Dann gäbe es folglich ein Strategieprofil $x^{n+1} \in X$, welches diesen Pfad in $\Gamma$ verlängert. Aber wie wir gerade gezeigt haben wäre dann auch $\phi(x^0), \dots, \phi(x^n), \phi(x^{n+1})$ ein Verbesserungspfad in $\Delta$, insbesondere also eine Verlängerung des ursprünglichen Pfades - im Widerspruch zu dessen vorausgesetzter Abgeschlossenheit.
\end{proof}

\begin{kor}\label{kor:ReflektNG}
	Ordinale Morphismen mit injektiver Spielerabbildung reflektieren Nash-Gleichgewichte. Das heißt, ist $x \in X$ ein Strategieprofil in $\Gamma$ und $(\sigma, \phi)$ ein ordinaler Morphismus in ein Spiel $\Delta$, sodass $\phi(x)$ ein Nash-Gleichgewicht in diesem ist, dann war bereits $x$ ein Nash-Gleichgewicht von $\Gamma$.
\end{kor}

\begin{proof}
	Es ist $x$ ein trivialer Verbesserungspfad in $\Gamma$, dessen Bild $\phi(x)$ in $\Delta$ abgeschlossen ist. Daher ist mit \Cref{prop:ordMorphVerbPf}  $x$ in $\Gamma$ abgeschlossen und folglich (vgl. \Cref{beob:VerbPfadeundNGe}) $x$ ein Nash-Gleichgewicht in $\Gamma$.
\end{proof}

\begin{prop}\label{prop:NGerhalten}
	Sei $(\sigma, \phi): \Gamma \to \Delta$ eine biordinale Retraktion (d.h. ein biordinaler Morphismus, welcher ein Rechtsinverses besitzt). Dann ist ein Strategieprofil $x \in X$ genau dann ein Nash-Gleichgewicht in $\Gamma$, wenn $\phi(x)$ ein Nash-Gleichgewicht in $\Delta$ ist.
\end{prop}

\begin{proof}
	Sei $(\tau, \psi): \Delta \to \Gamma$ ein Rechtsinverses von $(\sigma, \phi)$, d.h. es gelte $(\sigma, \phi) \circ (\tau, \psi) = (\tau\circ\sigma, \phi\circ\psi) = (\id_J, id_Y)$. Damit ist $\sigma$ offenbar injektiv und daher mit \Cref{kor:ReflektNG} die \glqq wenn\grqq-Richtung gezeigt.
	
	Zu \glqq dann\grqq: Sei also $x \in X$ ein Nash-Gleichgewicht und $y_j \in Y_j$. Wegen der Surjektivität von $\phi_j$ gibt es dann ein $\hat{x}_{\sigma(j)} \in X_{\sigma(j)}$ mit $\phi_j(\hat{x}_{\sigma(j)} = y_j)$. Da $x$ ein Nash-Gleichgewicht ist, gilt nun 
	\[c_{\sigma(j)}(x) \leq c_{\sigma(j)}(x \mid \hat{x}_{\sigma(j)})\]
	und daher wegen Biordinalität von $\phi$ auch 
	\[d_j(\phi(x)) \leq d_j(\phi(x \mid \hat{x}_{\sigma(j)})) \overset{\ast}{=} d_j(\phi(x) \mid \phi_j(\hat{x}_{\sigma(j)})) = d_j(\phi(x) \mid y_j)\]
	wobei in $\ast$ nochmal die Injektivität von $\sigma$ eingeht.
\end{proof}

\todo[inline]{In der Proposition explizit Surjektivität/Injektivität fordern - folgt die Retrakteigenschaft dann? In dem Fall also Bemerkung hinzufügen}

\begin{bsp}
	\todo{Bi?}Ordinale Isomorphismen haben insbesondere injektive Spieler- und Strategieabbildungen. Damit erfüllen sie die Voraussetzungen von \Cref{prop:NGerhalten}, d.h. ordinal isomorphe Spiele haben die gleichen Nash-Gleichgewichte.
\end{bsp}

\begin{bsp}
	Spieler mit nur einer einzigen Strategie können immer entfernt werden ohne die Nash-Gleichgewichte zu verändern. \todo{Formalisiert oBdA-Bemerkung aus Grundlagen}
	
	\todo[inline]{Beweis: Morphismus angeben und Eigenschaften zeigen}
\end{bsp}

\begin{prop}\label{prop:ordMorphVerbPf}
	Sei $\phi: \Gamma \to \Gamma'$ ein ordinaler Morphismus und $x^0, x^1, \dots$ ein Verbesserungspfad in $\Gamma$. Dann ist $\phi(x^0), \phi(x^1), \dots$ ein Verbesserungspfad in $\Gamma'$. Ist der Pfad endlich und in $\Gamma'$ abgeschlossen, so auch in $\Gamma$.
\end{prop}

\begin{proof}
	Da Morphismen spielerweise definiert sind, erhalten sie die Eigenschaft der unilateralen Abweichung. Die Ordinalität des Morphismus stellt ferner sicher, dass jeder Verbesserungsschritt ein Verbesserungsschritt bleibt.
	
	Nehmen wir nun an $\phi(x^0), \dots, \phi(x^n)$ wäre ein abgeschlossener Verbesserungspfad in $\Gamma'$, aber $x^0, \dots, x^n$ nicht abgeschlossen in $\Gamma$. Dann gäbe es folglich ein Strategieprofil $x^{n+1} \in X$, welches diesen Pfad in $\Gamma$ verlängert. Aber wie wir gerade gezeigt haben wäre dann auch $\phi(x^0), \dots, \phi(x^n), \phi(x^{n+1})$ ein Verbesserungspfad in $\Gamma'$, insbesondere also eine Verlängerung des ursprünglichen Pfades - im Widerspruch zu dessen vorausgesetzter Abgeschlossenheit.
\end{proof}

\begin{kor}
	Ordinale Morphismen reflektieren Nash-Gleichgewichte. Das heißt, ist $x \in X$ ein Strategieprofil in $\Gamma$ und $\phi$ ein ordinaler Morphismus in ein Spiel $\Gamma'$, sodass $\phi(x)$ ein Nash-Gleichgewicht in diesem ist, dann war bereits $x$ ein Nash-Gleichgewicht von $\Gamma$.
\end{kor}

\begin{proof}
	Es ist $x$ ein trivialer Verbesserungspfad in $\Gamma$, dessen Bild $\phi(x)$ in $\Gamma'$ abgeschlossen ist. Daher ist mit \Cref{prop:ordMorphVerbPf}  $x$ in $\Gamma$ abgeschlossen und folglich (vgl. \Cref{beob:VerbPfadeundNGe}) $x$ ein Nash-Gleichgewicht in $\Gamma$.
\end{proof}

\begin{kor}
	Seien $\Gamma$ und $\Gamma'$ zwei ordinal-isomorphe Spiele. Dann ist $x \in X$ genau dann ein Nash-Gleichgewicht von $\Gamma$, wenn $\phi(x) \in X'$ ein Nash-Gleichgewicht von $\Gamma'$ ist.
\end{kor}

Ordinal-isomorphe Spiele haben also die gleichen Nash-Gleichgewichte. Selbiges gilt auch für Verbesserungpfade, das heißt insbesondere, dass die FIP eines Spiels unter ordinaler Isomorphie erhalten bleibt.

\begin{kor}
	Seien $\Gamma$ und $\Gamma'$ zwei ordinal-isomorphe Spiele. Dann hat $\Gamma$ genau dann die FIP, wenn $\Gamma'$ diese besitzt.
\end{kor}


\begin{lemma}
	Seien $\Gamma$ und $\Gamma'$ zwei sozial isomorphe Spiele. Dann ist $x \in X$ genau dann ein soziales Optimum von $\Gamma$, wenn $\phi(x) \in X'$ ein soziales Optimum von $\Gamma'$ ist.
\end{lemma}

\begin{proof}
	.
	
	\todo[inline]{Folgt direkt mit Definitionen}
\end{proof}

\begin{satz}
	Besitzt ein Spiel $\Gamma$ ein ordinales Potential, so ist es ordinal-isomorph zu einem Auslastungsspiel.
\end{satz}

\begin{proof}
	Analog zum Beweis der Äquivalenz von Spielen mit exaktem Potential und Auslastungsspielen in \cite{MonShap}, Beweis orientiert sich an \cite{MultiPotGames}.
\end{proof}

\begin{beob}
	Besitzt ein Spiel ein verallgemeinertes ordinales Potential, so gibt es einen ordinalen Morphismus in/von \todo{Was von beidem?} ein Auslastungsspiel.
\end{beob}

\begin{proof}
	.
	\todo[inline]{Proofmining in oberem Beweis}
\end{proof}

\begin{beob}
	Nach \cite{MonShap} Lemma 2.5 hat jedes Spiel mit FIP ein verallgemeinertes Potential, also \todo[inline]{in/von...}
\end{beob}


Eine Verallgemeinerung von \Cref{beob:ZshExGewPot} lautet

\begin{prop}
	Seien $\Gamma$ und $\Delta$ zwei Spiele. Dann gibt es genau dann einen gewichteten Morphismus $\Gamma \to \Delta$, wenn es Gewichte $(w_i)_{i\in I}$ und einen exakten Morphismus $\Gamma \to w_i \cdot \Delta$ gibt.
\end{prop}

\begin{kor}
	Zwei Spiele $\Gamma$ und $\Delta$ sind genau dann gewichtet isomorph, wenn ... \todo{usw.}
\end{kor}