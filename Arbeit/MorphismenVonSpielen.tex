\section[Morphismen]{Morphismen und Isomorphismen von Spielen}\label{sec:Morphismen}

\todo[inline]{Motivation: Warum betrachtet man überhaupt Morphismen von Spielen. Wie induzieren diese Isomorphismen?}

\subsection{Definitionen}

Zu zwei gegebenen Spielen $\Gamma = (I, X, (K_i)_{i\in I}, (c_i)_{i\in I})$ und $\Gamma' = (I', X', (K_i)_{i\in I}, (c'_i)_{i\in I})$ kann man wie folgt eine Abbildung zwischen diesen beiden definieren:
\begin{itemize}
	\item Eine bijektive Abbildung $\sigma: I \to I'$ zwischen den Spielermengen und
	\item für jeden Spieler $i \in I$ eine Abbildung $\phi_i: X_i \to X'_{\sigma(i)}$ seiner Strategien.
\end{itemize}
\cite{Polyequilibrium} bezeichnet derartige Abbildungen \emph{Strategieersetzungsvorschriften}.\todo{Mehr dazu schreiben - erst später bei rationalen SEVs erwähnen?}

\begin{beob}
	Mit dieser Definition ist es also nur möglich Abbildungen zwischen Spielen mit Spielermengen gleicher Kardinalität zu definieren. Im Folgenden werden wir auch nur solche Abbildungen betrachten (vergleiche aber \Cref{bem:LapMorDef} dazu wie auch in anderen Fällen Abbildungen zwischen Spielen definiert werden können). Zur Vereinfachung der Notation werden wir daher ab sofort immer davon ausgehen, dass die Spielermengen beider an einer Abbildung beteiligten Spiele bereits gleich und geeignet permutiert sind.
\end{beob}

Abbildungen der obigen Form nehmen noch keinerlei Rücksicht auf die Kostenfunktionen der jeweiligen Spiele. Da diese aber in der Regel die interessierenden Eigenschaften eines Spiels (wie beispielsweise Gleichgewichte) festlegen, werden derartige Abbildungen im Allgemeinen noch wenig Aussagen über die beteiligten Spiele ermöglichen. So besagt der durch diese Art Abbildungen induzierte Isomorphiebegriff bspw. nur, dass zwei Spiele Spieler- und Strategiemengen gleicher Kardinalität besitzen.

Echte Morphismen zwischen Spielen sollten folglich noch mehr der Struktur eines Spiels erhalten, insbesondere in irgendeiner Form \glqq verträglich\grqq{} mit den Kostenfunktionen sein. Je nach dem, welche Eigenschaften die Morphismen (und insbesondere die dadurch induzierten Isomorphismen) erhalten sollen, erhält man so unterschiedlich starke Einschränkungen daran, welche Abbildungen zwischen Spielen als \emph{Morphismen zwischen Spielen} bezeichnet werden dürfen. Einige Möglichkeiten dafür werden wir nun kennenlernen:

\begin{defn}
	Zwei Spiele $\Gamma = (I, X, (K_i)_{i\in I}, (c_i)_{i\in I})$ und $\Gamma' = (I, X', (K_i)_{i\in I}, (c'_i)_{i\in I})$ heißen \emph{äquivalent}, wenn es für jeden Spieler $i$ eine bijektive Abbildung $\phi_i: X_i \to X'_i$ gibt, sodass gilt:
		\[\forall x \in X: c_i(x) = c'_i(\phi(x)) \]
	\todo{Permutation von Spielern erlauben? Das entspricht dann einer bereits von Nash verwendeten Definition!}
\end{defn}

\begin{bem}
	Zwei Spiele sind also genau dann äquivalent, wenn sie sich ausschließlich durch Umbenennung der Strategien ineinander überführen lassen. In \cite{MonShap} (S. 133) wird dies als Isomorphie von Spielen bezeichnet.
\end{bem}

\begin{defn}\label{def:SpielIsomLap}
	Zwei Spiele $\Gamma = (I, X, (K_i)_{i\in I}, (c_i)_{i\in I})$ und $\Gamma' = (I, X', (K'_i)_{i\in I}, (c'_i)_{i\in I})$ heißen \emph{isomorph}, falls es bijektive Abbildungen $\phi_i: X_i \to X'_i$ sowie bijektive und monotone Abbildungen $\psi_i: K_i \to K'_i$ gibt, sodass alle Diagramme der folgenden Form kommutieren:
	
	\begin{center}
		\begin{tikzcd}
			X \rar{\phi} \dar{c_i} & X' \dar{c'_i} \\
			K_i \rar{\psi_i}		& K'_i
		\end{tikzcd}
	\end{center}
\end{defn}

\begin{bem}\label{bem:LapMorDef}
	Diese Definition ergibt sich aus der abstrakteren Definition für \todo{...} in \cite{LapGameCat}. 
	
	\todo[inline]{Auch auf Verallgemeinerung mit Garben hinweisen (erlaubt Abbildungen zwischen Spielen mit Spielermengen unterschiedlicher Kardinalität!)}
\end{bem}

\begin{defn}
	Zwei im Sinne von \Cref{def:SpielIsomLap} isomorphe Spiele heißen \emph{sozial isomorph}, wenn zusätzlich die Funktion
		\[\sum \psi_i: \prod_{i \in I}K_i \to \prod_{i \in I} K'_i \]
	monoton ist. \todo{Das macht natürlich nur Sinn, wenn auf den beiden Produkträumen auch totale (?) Ordnungen existieren}
\end{defn}

\begin{bsp}
	lineare Funktionen
\end{bsp}	

\begin{defn}\label{def:NashMorphismus}
	Ein \emph{Nash-Morphismus} $\gamma: \Gamma \to \Gamma'$ zwischen zwei Spielen $\Gamma = (I, X, (K_i)_{i\in I}, (c_i)_{i\in I})$ und $\Gamma' = (I, X', (K'_i)_{i\in I}, (c'_i)_{i\in I})$ ist gegeben durch Abbildungen $\phi_i: X_i \to X'_i$ sodass gilt:
		\[\forall x\in X, i \in I, \hat{x}_i \in X_i: c_i(\hat{x}_i,x_{-i}) > c_i(x) \Rightarrow c'_i(\phi(\hat{x}_i,x_{-i})) > c'_i(\phi(x))\]
	Der Morphismus $\gamma$ heißt \emph{Nash-Isomorphismus} (und die beiden Spiele dann \emph{Nash-isomorph}), wenn die $\phi_i$ bijektiv sind und gilt:
		\[\forall x\in X, i \in I, \hat{x}_i \in X_i: c_i(\hat{x}_i,x_{-i}) > c_i(x) \iff c'_i(\phi(\hat{x}_i,x_{-i})) > c'_i(\phi(x))\]
\end{defn}

\begin{beob}
	Es gilt:
	\begin{center}
		äquivalent $\Rightarrow$ sozial isomorph $\Rightarrow$ isomorph $\Rightarrow$ Nash-isomorph
	\end{center}
\end{beob}


\subsection{Erste Sätze}

\begin{lemma}
	Seien $\Gamma$ und $\Gamma'$ zwei Nash-isomorphe Spiele. Dann ist $x \in X$ genau dann ein Nashgleichgewicht von $\Gamma$, wenn $\phi(x) \in X'$ ein Nashgleichgewicht von $\Gamma'$ ist.
\end{lemma}

\begin{proof}.
	
	\todo[inline]{folgt direkt mit Definitionen}
\end{proof}

\begin{lemma}
	Seien $\Gamma$ und $\Gamma'$ zwei Nash-isomorphe Spiele. Dann hat $\Gamma$ genau dann die FIP, wenn $\Gamma'$ diese besitzt.
\end{lemma}

\begin{proof}.
	
	\todo[inline]{Beweis über Verbesserungspfad im einen entspricht Verbesserungspfad im anderen. Evtl. direkt das als Lemma formulieren und dann die beiden vorherigen Lemmas als Korollare daraus?}
\end{proof}

\begin{lemma}
	Sei $\gamma: \Gamma \to \Gamma'$ ein Nash-Morphismus und $x \in X$. Ist dann $\phi(x) \in X'$ ein Nashgleichgewicht von $\Gamma'$, so ist auch $x$ selbst schon ein Nashgleichgewicht (von $\Gamma$).
\end{lemma}

\begin{proof}
	.
	
	\todo[inline]{Nachrechnen - evtl. mit vorherigen Sätzen verbinden bzw. schon davor zeigen, damit diese ein Korollar werden?}
\end{proof}

\begin{lemma}
	Seien $\Gamma$ und $\Gamma'$ zwei sozial isomorphe Spiele. Dann ist $x \in X$ genau dann ein soziales Optimum von $\Gamma$, wenn $\phi(x) \in X'$ ein soziales Optimum von $\Gamma'$ ist.
\end{lemma}

\begin{proof}
	.
	
	\todo[inline]{Folgt direkt mit Definitionen}
\end{proof}

\begin{satz}
	Besitzt ein Spiel $\Gamma$ ein ordinales Potential, so ist es isomorph zu einem Auslastungsspiel.
\end{satz}

\begin{proof}
	Analog zum Beweis der Äquivalenz von Spielen mit exaktem Potential und Auslastungsspielen in \cite{MonShap}, Beweis orientiert sich an \cite{MultiPotGames}.
\end{proof}

\begin{beob}
	Besitzt ein Spiel ein verallgemeinertes ordinales Potential, so gibt es einen Nash-Morphismus in/von \todo{Was von beidem?} ein Auslastungsspiel.
\end{beob}

\begin{proof}
	.
	\todo[inline]{Proofmining in oberem Beweis}
\end{proof}

\begin{beob}
	Nach \cite{MonShap} Lemma 2.5 hat jedes Spiel mit FIP ein verallgemeinertes Potential, also \todo[inline]{in/von...}
\end{beob}