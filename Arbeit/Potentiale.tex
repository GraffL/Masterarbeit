\section{Potentiale}\label{sec:Potentiale}

\subsection{Definitionen}

\begin{defn}
	Eine Funktion $P: X \to \IR$ heißt\todo{Verallgemeinern? (total geordnete Menge $K$ statt $\IR$?)}
	\begin{itemize}
		\item \emph{Beste Antwort-Potential}, wenn für jeden Spieler $i$ und alle Strategieprofile $x_{-i} \in X_{-i}$ gilt:
			\[\arg \min_{x_i \in X_i}c_i(x) = \arg \min_{x_i \in X_i} P(x)\]
		\item \emph{verallgemeinertes ordinales Potential}, wenn für jeden Spieler $i$ und alle Strategieprofile $x_{-i} \in X_{-i}$ sowie $x_i, \hat{x}_i \in X_i$ gilt:
			\[c_i(x_i,x_{-i}) > c_i(\hat{x}_i, x_{-i}) \implies P(x_i,x_{-i}) > P(\hat{x}_i, x_{-i})\]
		\item \emph{ordinales Potential}, wenn für jeden Spieler $i$ und alle Strategieprofile $x_{-i} \in X_{-i}$ sowie $x_i, \hat{x}_i \in X_i$ gilt:
			\[c_i(x_i,x_{-i}) > c_i(\hat{x}_i, x_{-i}) \iff P(x_i,x_{-i}) > P(\hat{x}_i, x_{-i})\]
	\end{itemize}
	Sind die $K_i$ zudem geordnete abelsche Gruppen, so heißt $P$
	\begin{itemize}
		\item \emph{skaliertes Potential}, wenn es streng monotone Funktionen $f_i: \IR \to K_i$ gibt, sodass für jeden Spieler $i$ und alle Strategieprofile $x_{-i} \in X_{-i}$ sowie $x_i, \hat{x}_i \in X_i$ gilt:
			\[c_i(x_i,x_{-i}) - c_i(\hat{x}_i, x_{-i}) = f_i(P(x_i,x_{-i}) - P(\hat{x}_i, x_{-i}))\]
	\end{itemize}
	Sind die $K_i$ Teilmengen eines gemeinsamen geordneten Rings $K$, so heißt $P$
	\begin{itemize}	
		\item \emph{gewichtetes Potential}, wenn es einen Gewichtsvektor $(w_i)_{i\in I}$ gibt, sodass für jeden Spieler $i$ und alle Strategieprofile $x_{-i} \in X_{-i}$ sowie $x_i, \hat{x}_i \in X_i$ gilt:
			\[c_i(x_i,x_{-i}) - c_i(\hat{x}_i, x_{-i}) = w_i\cdot(P(x_i,x_{-i}) - P(\hat{x}_i, x_{-i}))\]
		\item \emph{exaktes Potential}, wenn für jeden Spieler $i$ und alle Strategieprofile $x_{-i} \in X_{-i}$ sowie $x_i, \hat{x}_i \in X_i$ gilt:
			\[c_i(x_i,x_{-i}) - c_i(\hat{x}_i, x_{-i}) = P(x_i,x_{-i}) - P(\hat{x}_i, x_{-i})\]
	\end{itemize}
\end{defn}\todo{Kann man diese Definition irgendwie kompakter/übersichtlicher machen?}

Exakte, gewichtete, ordinale und verallgemeinerte ordinale Potentiale wurden erstmals in \cite{MonShap} definiert, beste Antwort-Potentiale erstmals in \cite{BestRespPot}.

\todo[inline]{Anschauung}

\todo[inline]{Zusammenhänge (evtl. schon nächstes Kapitel?)}

\subsection{Erste Sätze}

Zu einem gegebenen Strategieprofil $x \in X$ sei dessen \emph{Nachbarschaft} die Menge aller durch höchstens eine Abweichung erreichbarer Strategieprofile, d.h. die Menge $\{(\hat{x}_i, x_{-i}) | i \in N, \hat{x}_i \in X_i\}$. Wir nennen $x$ dann ein \emph{lokales Minimum} einer Funktion $f: X \to \IR$, wenn es ein Minimum innerhalb seiner Nachbarschaft ist.

\begin{satz}\label{satz:lokMinNG}
	Sei $\Gamma$ ein Spiel mit einem verallgemeinerten ordinalen Potential $P$. Dann ist jedes lokale Minimum von $P$ ein Nash-Gleichgewicht von $\Gamma$. Ist $P$ sogar ein ordinales Potential, so gilt auch die umgekehrte Richtung.
\end{satz}

\todo[inline]{Diese Sätze hier evtl. nur erwähnen und erst später (nach der Definition von Morphismen) formalisieren (und beweisen)}

Dieser Satz zeigt also, dass man Nash-Gleichgewichte allein durch Betrachten einer Potentialfunktion finden kann. Insbesondere folgt daraus direkt die Existenz von Nash-Gleichgewichten in einer Vielzahl von Potentialspielen:

\begin{kor}
	Sei $\Gamma$ ein Spiel mit einem kompakten Strategieraum und einer stetigen verallgemeinerten ordinalen Potentialfunktion. Dann hat $\Gamma$ wenigstens ein Nash-Gleichgewicht.
\end{kor}

Insbesondere also haben endliche Potentialspiele immer ein Nashgleichgewicht. \Cref{satz:lokMinNG} folgt mit Hilfe von \Cref{kor:ExVerbPfadExNG} direkt aus dem folgenden Satz\todo{Streng genommen nicht wirklich, da das Korollar nur für Spiele gilt!?}:

\begin{satz}
	Sei $\Gamma$ ein Spiel mit einem verallgemeinerten ordinalen Potential $P$. Dann ist jeder Verbesserungspfad in $\Gamma$ auch ein Verbesserungspfad bezüglich $P$. Ist $P$ sogar ein ordinales Potential, so gilt auch die umgekehrte Richtung.
\end{satz}

\begin{proof}.
	
	\todo[inline]{Beweis}	
\end{proof}

\todo[inline]{Was kann man über Beste-Antwort-Potentiale sagen (vermtl. Zusammenhang zu Beste-Antwort-Pfade?)}

\begin{satz}
	Hat ein Spiel die FIP, so besitzt es auch ein verallgemeinertes ordinales Potential.
	
	Umgekehrt besitzt jedes \emph{endliche} Spiel mit einem verallgemeinerten ordinalen Potential auch die FIP.
\end{satz}

\begin{proof}
	\cite{MonShap}/\cite{CongGamesPlayerSpecPayoff} (konstruktiver Beweis)
\end{proof}